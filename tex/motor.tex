\فصل{کنترل کننده موتور}
\برچسب{chap:motor}

یک کنترل کننده موتور یک دستگاه الکترونیکی است (معمولا به شکل یه مدار بدون پوشش و محفظه است) که به عنوان یک دستگاه واسطه بین میکروکنترلر، یک منبع تغذیه یا باتری و موتورها عمل می کند.
اگرچه میکروکنترلر (مغز ربات) سرعت و جهت موتور را مشخص میکند، اما به دلیل محدودیت زیاد در تغذیه خروجی (جریان و ولتاژ) نمی تواند آن ها را مستقیما هدایت کند. از طرف دیگر درایور موتور میتواند جریان را در ولتاژ مورد نظر فراهم کند اما نمیتواند تصمیم بگیرد که موتور تا چه میزان سریع بچرخد.
بنابراین، میکروکنترلر و کنترل کننده موتور باید باهم کار کنند تا موتور به طور مناسبی حرکت کند. معمولا میکروکنترلر میتواند از طریق یک روش ارتباطی ساده مانند UART(serial) یا PWM به کنترل کننده موتور دستوالعملی برای چگونگی تغذیه موتور بدهد. هم چنین برخی از کنترل کننده های موتور را می توان به صورت دستی و با استفاده از ولتاژ آنالوگ کنترل کرد (معمولا با یک پتانسیومتر ایجاد می شود).
اندازه و وزن فیزیکی کنترل کننده موتور می تواند بسیار متفاوت باشد، از یک دستگاه کوچکتر از نوک انگشتتان برای کنترل یک ربات کوچک sumo تا یک کنترل کننده سنگین وزن چند کلیوگرمی. اندازه و وزن کنترل کننده ربات معمولا کمینه تاثیر را روی ربات دارد، تا زمانی که شما با ربات‌های خیلی کوچک و یا هواپیماهای بدون سرنشین کار می‌کنید که در این حالت کوچکترین وزن‌ها هم تاثیرگذار خواهد بود. اندازه کنترل کننده موتور معمولا به حداکثر جریانی که میتواند فراهم کند وابسته است. جریان بیشتر به معنی استفاده از سیم هایی با قطر بزرگتر است.
انواع کنترل کننده های موتور
از آنجا که انواع مختلفی از محرک ها یا عملگرها وجود دارد، انواع گوناگونی از کنترل کننده های موتور نیز وجود دارد:
کنترل کننده موتور های براش DC (با جاروبک): این کنترل‌کننده‌ها مختص موتورهای DC، موتورهای DC گیربکس‌دار و بیشتر موتورها یا عملگرهای خطی است.
کنترل کننده موتور های بدون جاروبک DC (موتورهای براش‌لس): موتور های بدون جاروبک DC از این کنترل کننده‌ها استفاده می کنند.
کنترل کننده موتوره ای Servo: که مورد استفاده سروو موتورها است.
کنترل کننده موتور های پله ای: موتور های تک قطبی یا دوقطبی پله ای بسته به نوع آن ها از این استفاده می کنند.
انتخاب یک کنترل کننده موتور
کنترل کننده موتور میتواند بلافاصله پس از انتخاب موتور یا محرک انتخاب شود. همچنین، جریان یک موتور به میزان گشتاوری که میتواند فراهم کند مرتبط است: یک موتور DC کوچک، مقدار جریان زیادی مصرف نمی کند، اما گشتاور زیادی هم فراهم نمی کند، در حالیکه یک موتور بزرگ میتواند گشتاور زیادی را ایجاد کند اما برای این کار به جریان زیادی نیاز دارد.

\قسمت{کنترل کننده موتور \چر{DC}}

اولین توجه ولتاژ نامی موتور است. کنترل کننده های موتور DC علاقه منداند که یک محدوده از ولتاژ را ارائه دهند. برای مثال، اگر موتور شما در ولتاژ نامی 3 ولت عمل می کند، شما نباید کنترل کننده موتوری را انتخاب کنید که تنها در محدوده 6 تا 9 ولت موتور را کنترل می کند. این به شما کمک می کند تا برخی از کنترل کننده های موتور را از لیست جدا کنید و انتخاب راحت تری داشته باشید.
زمانی که شما تعدادی از کنترل کننده های موتور را که در محدوده مناسب ولتاژ موتور را کنترل می کنند یافتید، توجه بعدی جریان مداومی است که کنترل کننده نیاز دارد تامین کند. شما باید کنترل کننده موتوری را پیدا کنید که جریان را برابر یا بالاتر از مصرف جریان مداوم موتور تحت بار ارائه دهد. اگر شما یک کنترل کننده موتور 5 آمپری را برای یک موتور 3 آمپری انتخاب کنید، موتور ها فقط به اندازه ای جریان میگیرند که به آن نیاز دارند. از طرف دیگر، یک موتور 5 آمپری به احتمال زیاد یک کنترل کننده موتور 3 آمپری را میسوزاند. بسیاری از کارخانه های موتور سازی موتور های DC جریان Stall موتور را مشخص می کنند که به شما دید روشنی نمیدهد که چه نوع کنترل کننده موتوری نیاز دارید. اگر شما نمیتوانید جریان مداوم موتور را پیدا کنید، یک قاعده ساده این است که جریان مداوم را حدود 20 تا 25 درصد جریان Stall تخمین بزنید. همه کنترل کننده های موتور DC یک نرخ جریان بیشینه ارائه می کنند- اطمینان حاصل کنید که این نرخ حدودا دوبرابر جریان کارکرد مداوم موتور است. توجه کنید زمانی که موتور نیاز به تولید گشتاور بیشتری دارد (به عنوان مثال برای بالا رفتن از شیب) به جریان بیشتری نیاز دارد. انتخاب یک کنترل کننده موتور با جریانی بیش از حد و حفاظت حرارتی انتخاب بسیار خوبی است.
روش کنترل، دیگر توجه مهم است. روش های کنترل شامل ولتاژ آنالوگ، 12C، PWM، R/C، UART(serial) هستند. اگر شما از یک میکروکنترلر استفاده می کنید بررسی کنید که کدام پین ها موجود هستند و کدام موتور برای انتخاب شما مناسب است. اگر میکروکنترلر شما پین های ارتباطی سری دارد، شما میتوانید کنترل کننده موتور سری انتخاب کنید؛ برای PWM شما احتمالا به یک کانال PWM برای هر موتور نیاز دارید.
توجه نهایی یک نکته تجربی است: درایور موتور تک کاناله در مقابل دو کاناله (Single vs. dual motor controller). یک کنترل کننده موتور دوکاناله DC میتواند سرعت و جهت دو موتور DC را مستقلا کنترل کند و اغلب موجب صرفه جویی در هزینه و زمان شما می شود. نیازی نیست که موتور ها یکسان باشد، اگرچه برای یک ربات متحرک، موتورهای درایو در اکثر موارد باید یکسان باشند. شما باید کنترل کننده موتور دوکاناله را بر اساس قوی ترین موتور Dc انتخاب کنید. توجه داشته باشید که کنترل کننده موتور دوکاناله تنها یک ورودی تغذیه دارند، بنابراین، اگر شما میخواهید یک موتور را در 6 ولت و دیگری را در 12 ولت کنترل کنید، این کار غیر ممکن است. توجه داشته باشید که نرخ جریان فراهم شده تقریبا همیشه بر کانال یعنی برای هر کانال بیان می‌شود که از دیتا شیت کنترل کننده انتخابی تعیین می‌گردد.


\قسمت{راه اندازی و کنترل سرعت موتور \چر{DC}}

سرعت موتور DC به مجموعه ای از ولتاژ و جریان عبوری از سیم پیچ های موتور و نیز بار و گشتاور موتور، بستگی دارد. لذا ساده ترین راه برای کنترل سرعت موتور DC استفاده از مقاومت متغییر مطابق شکل زیر است.

از روش های پرکاربرد دیگری که برای کنترل سرعت موتورهای DC استفاده می شود می توان به مدولاسیون عرض پالس اشاره کرد. از تکنولوژی مدولاسیون عرض پالس(PWM ) برای کنترل توان استفاده می شود. در این روش از پالس های ولتاژ مربعی برای تغذیه موتور استفاده شده، که مقدار توان اعمالی در آن به مقدار dutycycle وابسته است.

همان طور که در شکل می بینید، dutycycle برابر است با نسبت مدت زمانی که سیگنال در وضعیت high قرار دارد، به کل دوره تناوب. نحوه کنترل سرعت موتور به وسیله مدولاسیون عرض پالس به این صورت است که ابتدا یک فرکانس ثابت و مناسب انتخاب شده و سپس برای افزایش سرعت موتور مقدار dutycycle افزایش و برای کاهش سرعت موتور مقدار dutycycle کاهش می یابد. فرکانس پالس PWM مناسب برای راه اندازی موتور DC بسته به نوع موتور معمولا بین 1 تا 100 کیلوهرتز انتخاب می شود. از طرفی معمولا پالس های PWM توسط میکرو کنترلرها تولید می شوند، و با توجه به اینکه خروجی میکروکنترلرها معمولا دارای ولتاژ 5 ولت و جریان در حد میلی آمپر می باشند، لذا قادر به تامین توان لازم برای راه اندازی این تجهیزات نمی باشد. در این صورت مدارات درایور، ولتاژ و جریان لازم را فراهم می کنند.
درایو یک طرفه:
در این روش می توان از طریق کنترل میزان dutycycle پالس های PWM که به بیس ترانزیستور اعمال می شوند، سرعت موتور تنها در یک جهت کنترل شود. برای موتورهای کوچک از ترانزیستور BJT به صورت شکل زیر استفاده می شود.

قطعات و تجهیزاتی که در آنها بار سلفی وجود دارد مانند رله، موتورهای سنولوئید، DC و... در هنگام قطع و وصل جریان بار، به خاطر خاصیت خود القایی، اثر تخلیه یا لگد القایی در آنها به وجود می آید که می تواند باعث آسیب رساندن به قطعات مدار و در اینجا به ترانزیستور شود. اثر لگد القایی به این صورت است، زمانی که کلید وصل می شود، جریان از VCC به سمت GND  می گذرد و هنگامی که کلید قطع می شود به خاطر خاصیت خودالقایی سلف، ولتاژی در جهت عکس جریان اولیه ایجاد شده و جریان متناسب با آن می خواهد در جهت عکس از ترانزیستور عبور کند. لذا برای حل این مشکل از دیود هرزگرد مطابق تصویر بالا استفاده می شود. حضور دیود هرزگرد باعث می شود که جریان به جای ترانزیستور از آن عبور کرده و دوبار وارد موتور شود. به این ترتیب این جریان داخی خود موتور از بین می رود.
همچنین برای بارهای نیازمند جریان بیشتر، به جای ترانزیستور BJT، می توان از ترازیستورهای دارلینگتون برای جریان هایی تا یک آمپر و از ترانزیستور MOSFET برای جریان هایی تا چندین آمپر و برای بارهایی با جریان بالاتر از ترانزیستور IGBT استفاده کرد.
درایور دو طرفه پل:
درایورهای یک طرفه فقط موتور را در یک جهت خاموش و روشن می کنند. برای حرکت موتور در دو جهت باید قطب های آن تغییر کند. برای این منظور از درایورهایی با چهار ترانزیستور موسوم به H یا H bridge استفاده می شود.
معرفی پایه های (Pinout) ماژول درایور دو کاناله L9110S
		 				 				
				
								
				
ماژول‌های درایور موتور دو کاناله L9110S دارای پایه‌هایی به شرح زیر هستند:
VCC: ولتاژ تغذیه ماژول
GND: زمین
M-A: پایه موتور A
M-B: پایه موتور B
A-1: سیگنال کنترلی موتور A
A-2: سیگنال کنترلی موتور A
B-1: سیگنال کنترلی موتور B
B-2: سیگنال کنترلی موتور B
پین اوت (Pinout) دو ماژول زیر که هر دو بر مبنای آی‌سی L9110S هستند، در تصاویر زیر مشخص شده‌اند.
					
						
				
			توجه شود که با حالت‌های مختلف سیگنال‌های کنترلی، عملکرد موتورها متفاوت می‌شود. برای مثال حالت‌های مختلف پایه‌های کنترلی موتور A (A-1 , A-2) با ورودی دیجیتال در تصویر زیر بیان شده‌اند.
		
				



