\فصل{ماژول‌های \واژه‌متن{wifi} مبتنی بر مدار مجتمع \چر{ESP8266}}
\برچسب{chap:esp}

ماژول‌های \واژه{wifi} جهت برقراری ارتباط با دستگاه‌های مبتنی بر وای فای مثل مودم‌ها روتر‌ها یا همان مسیریابها و دیگر تجهیزات مبتنی بر شبکه استفاده می‌شود. از این ماژول‌ها برای استفاده‌های مختلفی در پروژه‌های الکترونیکی استفاده می‌شود.

این ماژول‌ها در توان‌های مختلف و ابعاد مختلفی موجود هستند و تراشه‌های متفاوتی در آن‌ها به کار می‌رود؛ به صورت پیش فرض این ماژول‌ها برای استفاده در کنار یک پردازنده دیگر و یا یک وسیله کنترل کننده خارجی به کار می‌رود.

برخی ماژول‌های وای فای قابلیت استفاده از یک آنتن خارجی را نیز دارند ولی در همین حال یک آنتن مخصوص روی برد ماژول تعبیه گردیده است که به کاربر اجازه می‌دهد تا بدون نیاز به یک آنتن خارجی و با استفاده از فضای بسیار کم ماژول را استفاده کند. البته این انتخاب هزینه خود را دارد و موجب می‌شود که برد موثر ارتباط \واژه{wifi} کاهش یافته و برای برقراری ارتباط در فواصل بلند نیاز به مصرف توان بیشتری باشد\مرجع{ism-band}.

\قسمت{آنتن برد مدارچاپی}

آنتن برد مدارچاپی شامل مسیری رساناست که مستقیماً روی برد مدارچاپی کشیده شده است. علاوه بر این، بسته به نوع آنتن و فضای مورد نیاز شما، نوع مسیر متفاوت خواهد بود. نمونه‌هایی از انواع مسیر‌های مورد استفاده برای آنتن‌های برد مدارچاپی عبارتند از مسیرهای وارونه نوع F، مسیرهای مستقیم، مسیرهای مارپیچی، مسیرهای خمیده و یا مسیرهای دایره‌ای.

به طور کلی، هدف از آنتن مسیر در برد مدارچاپی ارائه روش ارتباط بی‌سیم است. علاوه بر این، در مرحله تولید برد مدارچاپی، مسیر روی سطح برد مدارچاپی لمینت می‌شود. با این حال، در موارد خاص، این مسیرها چندین لایه از برد مدارچاپی چند لایه را اشغال می‌کنند.

علاوه بر این، طول مسیر برد مدارچاپی فرکانس رزونانس را تعیین می‌کند و هرچه فرکانس بیشتر باشد، مسیر کوتاهتر است. همچنین برای هر آنتن روی برد مدار چاپی، یک منطقه نگهدارنده ضروری است. علاوه بر این، یک منطقه نگهدارنده یک منطقه خاص در اطراف آنتن برد مدارچاپی است که اجازه نمی‌دهد از پر شدن صفحه‌ی زمین زیر آنتن یا مسیرهای مسی روی هر لایه از برد مدارچاپی استفاده شود.

چون خود آنتن نیز باید امپدانس مشخصی داشته باشد معمولاً از بردهای دو لایه استفاده می‌شود تا بتوان مقدار امپدانس آن را در هنگام طراحی تعیین کرد و در صورت نیاز لایه‌ی \واژه{gplane} استفاده شود؛ ولی معمولاً لایه‌های زیرین یک آنتن پی سی بی را به صورت نارسانا درست می‌کنند تا نتایج دلخواه راحت‌تر به دست آید.

مواد داخلی بردهای مدار چاپی معمولاً از فایبرگلاس یا مواد مخصوصی که \واژه{fr4} نام دارد ساخته می‌شود. به هنگام طراحی هر آنتن باید مشخصات مواد استفاده شده در ساخت برد مدار چاپی نیز در نظر گرفته شود چون بر نتیجه نهایی و عملکرد آنتن بسیار موثر است.

\زیرزیرقسمت{مزایا}

آنتن‌های روی برد مدارچاپی معمولاً به علت جای کمتر و عدم نیاز به استفاده از قطعات خارجی و همچنین کاهش هزینه استفاده می‌شود. این نوع آنتن‌ها معمولاً یک نوار مسی است که در فرکانس خاصی عمل می‌کند و برای آن طراحی شده است. به همین جهت عملاً می‌توان از آنها به عنوان جایگاهی برای لحیم کردن قطعات دیگری هم استفاده کرد تا عملیات تطبیق امپدانس به کمک عناصر گسسته نیز روی همان سطح انجام شود.

آنتن‌های مدارچاپی معمولاً مزایای تولید انبوه را دارند و نتیجه آن کاهش هزینه‌های ساخت است\مرجع{pcb-pros-cons}..

\زیرزیرقسمت{معایب}

این آنتن‌ها معمولاً همیشه روی صفحات تخت هستند به همین علت محدودیت خاص خود را دارند و نمی‌توانند فرم‌های ۳ بعدی داشته‌باشند و همین عامل طراحی را بسیار محدود می‌کند. این آنتن‌ها پس از ساخت دیگر قابل تغییر نیستند و معمولاً قبل از اینکه نتیجه نهایی حاصل شود در چند مرحله آنتن را تنظیم می‌کنند تا نتیجه مطلوب حاصل شود ولی پس از ساخت نهایی برد، دیگر نیاز به تغییرات خاص آنتن وجود ندارد و همراه بقیه برد، به هنگام اسیدکاری و سرهم کردن برد مدار چاپی چند لابه ساخته می‌شود و مرحله جدا برای تولید نیاز ندارد.

همچنین، چون عملا این آنتن‌ها یک الگوی مشخص از نوارهای مسی هستند و می‌توانند به صورت دیجیتالی تغییر داده شوند، مناسب برای افراد تازه کار و کسانی که در حوزه‌های مخابراتی تجربه زیادی ندارند نیز هستند. یک طراح تازه کار می‌تواند با استفاده از نرم افزار مخصوص یکی از این آنتن‌ها را با مشخصات مورد نظر خود طراحی کرده و بدون داشتن دانش پیشرفته در این زمینه از آن در طرح خود استفاده کند\مرجع{pcb-pros-cons}.

\زیرزیرقسمت{آنتن مورد استفاده در ماژول ESP-01}

این آنتن از نوع میکرواستریپ بوده که برای تطبیق امپدانسی آن از یک \واژه{stub} و یک شبکه‌ی متشکل از \واژه{dpe} استفاده شده است. آنتن یاد شده، استاب و شبکه به ترتیب از سمت راست در شکل \رجوع{fig:esp01-top}‌ مشاهده می‌شوند.

\قسمت{انواع ماژول‌های \واژه‌متن{wifi}}

دو خانواده از معروف‌ترین مدارات مجتمع دارای ادوات \واژه{wifi} خانواده‌های \چر{\چر{ESP8266}} و \چر{\چر{ESP32}} هستند که هر دو محصول یک شرکت‌اند. در ادامه، هر کدام از این خانواده به صورت خلاصه معرفی می‌گردد.

\زیرقسمت{خانواده‌ی مبتنی بر \چر{ESP8266}}

\شروع{شکل}[ht]
\تنظیم‌ازوسط
\درج‌تصویر[پهنا=0.48\پهنای‌متن]{esp-01}
\شرح{ماژول \چر{ESP-01}. این ماژول قدیمی‌ترین و ساده‌ترین  ‌در میان ماژول‌های مبتنی بر این تراشه است\مرجع{esp-modules}}
\برچسب{fig:esp-01}
\پایان{شکل}

شرکت \واژه{espressif} یکی از شرکت‌های تولید کننده تراشه‌های مجهز به ادوات وای فای است. ماژول‌های متعددی بر اساس این پردازنده ساخته شده است که هر کدام مناسب شرایط خاصی هستند و به عنوان مثال ماژول \چر{ESP-01} یکی از ساده ترین گروه بوده و با داشتن امکانات کم و تعداد خروجی و ورودی اندک و یک تراشه حافظه ۵۱۲ کیلوبایتی (و بعضاً در نسخه‌های جدیدتر این ماژول حافظه‌های یک مگابایتی)، ارزان‌ترین نوع این ماژول‌ها است. تصویر این ماژول در شکل \رجوع{fig:esp-01} مشاهده می‌شود.

\شروع{شکل}[ht]
\تنظیم‌ازوسط
\درج‌تصویر[پهنا=0.48\پهنای‌متن]{esp-12f}
\شرح{ماژول \چر{ESP-12F}\مرجع{esp-modules}}
\برچسب{fig:esp-12f}
\پایان{شکل}

ماژول دیگری از این خانواده \چر{ESP-12F}‌ نام دارد که بیشترین تعداد پایه‌های ورودی و خروجی را در دسترس مصرف‌کننده قرار می‌دهد. به این ترتیب می‌توان از حداکثر قابلیت‌های تراشه استفاده کرد. این ماژول به همراه یک شیلد الکترومغناطیسی عرضه می‌شود تا از بروز تداخل الکترومغناطیسی جلوگیری شود؛ چه به منظور کاهش آلودگی رادیویی تولید شده توسط ماژول و چه برای کاهش تأثیرپذیری از امواج ناخواسته‌ی خارجی. تصویر این ماژول نیز در شکل \رجوع{fig:esp-12f} دیده می‌شود.

در این پروژه به علت نیاز به پایه‌های کم و جهت کاهش قیمت نهایی هر واحد استفاده از ماژول \چر{ESP-01} کاملا مقرون به صرفه بوده و تمام نیازهای پروژه را به خوبی برآورده می‌کند و در نتیجه برای ساخت از این ماژول استفاده شد. البته، برای ساخت نمونه‌ی اولیه و انجام آزمایش‌های مربوطه برای اطمینان از صحت عملکرد کنترلر بر روی برد آزمایشی، از ماژول \چر{ESP-12F}‌ استفاده شد تا در صورت نیاز از ورودی و خروجی برای اشکال‌یابی استفاده شود؛ اما محصول نهایی نیازی به این تعداد از پایه‌ها نداشته و ماژول ساده‌تر کاملاً برای اهداف پروژه کفایت می‌کند.

\زیرقسمت{خانواده‌ی مبتنی بر \چر{ESP32}}

خانواده دیگری از پردازنده‌های مجهز به وای فای این شرکت سری \چر{ESP32} است که نسل جدیدتری نسبت به \چر{ESP8266} محسوب شده و دارای امکانات بیشتری است.

این سری نسل کم هزینه و کم توان پردازنده‌های مجهز به وای فای است که از پروتکل\واژه{bt} نیز پشتیبانی می‌کند. هر تراشه دارای ۱یا ۲ پردازنده از خانواده \واژه{xtensa} محصول شرکت \واژه{tensilica} است. این تراشه مجهز به سوییچ آنتن، بالون \چر{RF}، تقویت کننده توان برای فرستادن امواج و تقویت کننده کم نویز مخصوص دریافت آن‌ها، فیلترها و ادوات مدیریت توان است.

\قسمت{تراشه \چر{ESP8266}}

همان طور که گفته شد، در این پروژه از ماژول \چر{ESP-12} استفاده شد که دارای یک تراشه‌ی \چر{ESP8266} است (شکل \رجوع{fig:esp8266}). این تراشه در واقع یک \واژه{mc} با توان پردازشی مناسب بوده که به ادوات \واژه{wifi} نیز مجهز است و قابلیت برنامه‌ریزی و اجرای برنامه‌های دلخواه کاربر را دارد و همچنین ابزارهای نرم‌افزاری لازم برای تهیه‌ی برنامه و برنامه‌نویسی این وسیله برای افرادی که توانایی کدنویسی را دارند نیز عرضه شده است. در ادامه، به معرفی توانایی‌های این میکروکنترلر می‌پردازیم.

\زیرقسمت{پردازنده‌ی \چر{ESP8266}}

\شروع{شکل}[ht]
\تنظیم‌ازوسط
\درج‌تصویر[پهنا=0.4\پهنای‌متن]{esp8266}
\شرح{دو تراشه‌ی \چر{ESP8266}‌ از نمای بالا (سمت راست) و پایین (سمت چپ).}
\برچسب{fig:esp8266}
\پایان{شکل}

این تراشه دارای یک پردازنده تک هسته ای با ۹۶ کیلو بایت حافظه دسترسی تصادفی است. فرکانس پردازنده در حالت پیشفرض برابر ۸۰ مگاهرتز بوده که در صورت لزوم و نیاز به پردازش در سرعت‌های بیشتر می‌توان آنرا به مقدار ۱۶۰ مگاهرتز تغییر داد. برای سیگنال ساعت از یک کریستال خارجی استفاده می‌شود که در برخی ماژول‌ها فرکانس آن برابر ۲۶ مگاهرتز و در برخی دیگر ۴۰ مگاهرتز است. این سیگنال سپس توسط یک واحد \واژه{pll} موجود در داخل تراشه به مقدار مورد نظر برای ساعت پردازنده (یعنی ۸۰ یا ۱۶۰ مگاهرتز) تبدیل می‌شود\مرجع{esp-ref}.

پردازنده‌ای که در این تراشه قرار دارد LX۱۰۶ نام دارد\مرجع{esp-ref}. امکانات پردازنده‌های مبتنی بر معماری اکستنسا بسیار قابل کنترل بوده و می‌توان در هنگام طراحی پردازنده ویژگی‌های مورد نیاز آن را فعال یا غیرفعال کرد. این ویژگی‌ها شامل انواع مختلف منابع وقفه، انواع دسترسی به ثبات‌ها و دیگر امکانات خاص این پردازنده است. در صورت روشن کردن یک ویژگی در هنگام طراحی نیاز است تا ترانزیستور‌های بیشتری برای پردازنده استفاده شده و طبیعتاً اندازه مساحتی که روی سطح سیلیکونی تراشه نهایی (شکل \رجوع{fig:die}) نیاز است افزایش می‌یابد و همچنین توان مصرفی سیستم نیز تحت تأثیر قرار می‌گیرد که ممکن است در وسایلی که با منابع تغذیه‌ی مستقلی مانند باتری کار می‌کنند باعث کاهش مدت زمانی که دستگاه می‌تواند بدون نیاز به تعویض باتری یا شارژ مجدد به کار خود ادامه دهد شود.

\شروع{شکل}[ht]
\تنظیم‌ازوسط
\درج‌تصویر[پهنا=0.4\پهنای‌متن]{die}
\شرح{تصویر گرفته‌شده از سطح سیلیکونی داخل تراشه.}
\برچسب{fig:die}
\پایان{شکل}

یکی از معایب این معماری، کمتر شناخته شده بودن آن است که موجب می‌شود پشتیبانی نرم افزاری برای تولید برنامه برای آن محدود باشد و طبیعتاً یافتن متخصصانی که در این زمینه توانایی تولید برنامه و برنامه نویسی داشته باشند سخت تر خواهد بود.

\زیرقسمت{ادوات موجود}

از آن‌جایی که این تراشه برای برقراری ارتباط با دیگر تراشه‌ها ممکن است به استفاده از پروتکل‌های ارتباطی مختلفی نیاز پیدا کند، تعدادی ادوات به صورت سخت‌افزاری در این تراشه پیاده‌سازی شده اند که علاوه بر کاهش حجم نرم‌افزار و کاهش هزینه‌های تولید برنامه، امکان برقراری ارتباط در سرعت‌های بالاتر از مقداری که مجموعه‌ای از دستورالعمل‌های پردازنده قادر به انجام آن هستند را فراهم می‌کند\مرجع{why-peripheral}.

مهم‌ترین این ادوات، مجموعه سخت‌افزار مرتبط با ارتباط \واژه{wifi} است که از چندین بخش مختلف تشکیل شده است و از ابزارهای کلیدی این تراشه به شمار می‌رود. دیگر ادواتی برای دسترسی به حافظه‌ی برنامه، کارت‌های حافظه‌ی جانبی، تقویت‌کننده‌های پخش صدای رقمی و همچنین اندازه‌گیری سطح ولتاژ یک سیگنال دلخواه وجود دارد\مرجع{esp-ref} که در ادامه به تفضیل به آن‌ها می‌پردازیم.

\زیرزیرقسمت{واحد \واژه{wifi}}

این واحد شامل چندین بخش است که برای ارسال و دریافت سیگنال‌های رادیویی با استفاده از آنتن و در صورت وجود شبکه‌ی تطبیق خارج از تراشه و مطابق استاندارد \چر{IEEE802.11}‌ به کار می‌رود. این مجموعه شامل یک دوسویه‌ساز، بالان، تقویت‌کننده با نویز کم، تقویت‌کننده توان \چر{RF} و شبکه‌ی تطبیق امپدانس داخلی است\مرجع{esp-ref}.

\واژه{deplexer} یا \واژه{trs}
، یک کلید الکترونیکی است که از آن به منظور اتصال یک آنتن مشترک به خروجی فرستنده و یا ورودی گیرنده استفاده می‌شود.

\واژه{balun}
یک ابزار الکترونیکی برای تبدیل سیگنال خطوط متوازن و نامتوازن به یکدیگر استفاده می‌شود. یک خط متوازن، یک خط انتقال است که هر دو رسانای مورد استفاده در آن از یک نوع هستند و هر کدام در طول خود امپدانس یکسانی نسبت به زمین و مدارات دیگر دارند. به طور مشابهی، خطوط نامتوازن خطوطی هستند که یکی از رساناها نسبت به زمین امپدانس مشخصه‌ی متفاوتی نسبت به دیگر رسانا (یا رساناها) دارد. کابل‌های هم محور\واژه{coax}، خطوط میکرواستریپ\واژه{ustrip} و انواعی از آنتن‌ها (مانند آنتنی که روی ماژول‌های \واژه{wifi} مورد استفاده قرار گرفته است) از این نوع خطوط هستند.

\glspl{lna}
 به عنوان اولین مرحله‌ی تقویت کنندگی در گیرنده‌های رادیویی استفاده می‌شود تا بتوان سیگنال‌های بسیار ضعیف را دریافت کرد. جهت نگهداری نسبت سیگنال به نویز در ارتباطات رادیویی این تقویت‌کننده بسیار نزدیک به آنتن ورودی قرار می‌گیرد و از عناصر کلیدی در هر گیرنده‌ی حساسی است. این تقویت‌کننده‌ها معمولاً بهره‌ی توان ۱۰۰ (۲۰ دسی‌بل) را می‌توانند تأمین کنند.

تقویت‌کننده‌ی توان \چر{RF} نوعی تقویت‌کننده سیگنال است که برای تقویت توان سیگنالی که قرار است ارسال شود به کار می‌رود و بین منبع تولید سیگنال نهایی و آنتن فرستنده قرار می‌گیرد.

\شروع{شکل}[ht]
\تنظیم‌ازوسط
\درج‌تصویر[trim={80pt 220pt 60pt 220pt},clip,پهنا=0.5\پهنای‌متن]{esp01-top}
\شرح{نمای بالا از ماژول \چر{ESP-01}. در سمت راست و در بالای کریستال ۲۶ مگاهرتزی، شبکه‌ی تطبیق امپدانس بین آنتن و تراشه دیده می‌شود.}
\برچسب{fig:esp01-top}
\پایان{شکل}

جهت انتقال حداکثر توان خروجی هر منبع سیگنال (از جمله خروجی تقویت کننده توان) لازم است امپدانس‌های منبع و بار با یکدیگر منطبق باشند تا تلفات ناشی از انتقال به کمترین میزان خود برسد. در صورتی که امپدانس بار متفاوت از امپدانس منبع تغذیه باشد شبکه‌ای در این میان قرار داده می‌شود که به آن شبکه‌ی تطبیق امپدانس گفته می‌شود. در ماژول \چر{ESP-01} نمونه‌ای از این شبکه که بین آنتن و تراشه‌ی اصلی قرار گرفته است و با تعدادی از عناصر فشرده‌ی خازنی و مقاومتی پیاده‌سازی شده است مشاهده می‌شود (شکل \رجوع{fig:esp01-top}).

\زیرزیرقسمت{واحد \واژه‌متن{qspi}}

رابط جانبی سریال (\واژه{spi}) یکی از پرکاربردترین رابط‌های بین میکروکنترلر و مدارهای مجتمع جانبی مانند حسگرها، مبدل‌های قیاسی به رقمی،مبدل‌های رقمی به قیاسی، انتقال‌دهنده‌های ثبّات، حافظه‌های دستیابی تصادفی ایستا و سایر موارد است. این مقاله شرح مختصری از رابط \واژه{spi} و پس از آن معرفی سوئیچ‌ها و ماکس‌های فعال دستگاههای آنالوگ و نحوه کمک آنها به کاهش تعداد ورودی-خروجی‌های دیجیتال در طراحی برد سیستم را ارائه می‌دهد.

\واژه{spi}
یک رابط همزمان و تمام-دوطرفه مبتنی بر رابط ارباب-برده است. داده‌های اصلی یا برده در لبه ساعت در حال افزایش یا سقوط همزمان می‌شوند. هم ارباب و هم برده می‌توانند داده‌ها را همزمان ارسال کنند. رابط \واژه{spi} می‌تواند 3 سیم یا 4 سیم باشد.

دستگاه‌های 4 سیم \واژه{spi} دارای چهار سیگنال هستند:

\شروع{فقرات}
\فقره ساعت (\واژه{sclk})
\فقره انتخاب تراشه (\واژه{cs})
\فقره خروجی ارباب، ورودی برده (\واژه{mosi})
\فقره ورودی ارباب، خروجی برده (\واژه{miso})
\پایان{فقرات}

دستگاهی که سیگنال ساعت را تولید می‌کند، ارباب نامیده می‌شود. داده‌های منتقل شده بین ارباب و برده با ساعت تولید شده توسط ارباب همگام‌سازی می‌شود. دستگاه‌های \واژه{spi} فرکانس‌های ساعت بسیار بیشتری را در مقایسه با رابط‌های I2C پشتیبانی می‌کنند. کاربران باید برگه اطلاعات محصول را برای مشخص کردن فرکانس ساعت رابط \واژه{spi} مطالعه کنند. رابط‌های \واژه{spi} می‌توانند فقط یک ارباب داشته باشند و می‌توانند یک یا چند برده داشته باشند.

سیگنال انتخاب تراشه از ارباب برای انتخاب تراشه‌ی برده استفاده می‌شود. این معمولاً یک سیگنال پایین-فعال است و برای قطع اتصال برده از گذرگاه \واژه{spi} به بالا کشیده می‌شود. هنگامی که از چندین برده استفاده می‌شود، یک سیگنال انتخاب تراشه جداگانه برای هر برده از ارباب لازم است. در این مقاله، سیگنال انتخاب تراشه همیشه یک سیگنال پایین-فعال است.

\واژه{mosi}
و \واژه{miso} خطوط داده هستند. \واژه{mosi} داده‌ها را از ارباب به برده و \واژه{miso} داده‌ها را از برده به ارباب منتقل می‌کند.

\شروع{شکل}[ht]
\تنظیم‌ازوسط
\درج‌تصویر[پهنا=\پهنای‌متن]{spi}
\شرح{نحوه‌ی کار پروتکل \واژه‌متن{spi} \مرجع{spi-intro}}
\برچسب{fig:spi}
\پایان{شکل}

برای شروع ارتباط \واژه{spi}، ارباب باید سیگنال ساعت را ارسال کرده و با فعال کردن سیگنال انتخاب برده، آن را انتخاب کند. همان طور که گفته شد، معمولاً انتخاب تراشه یک سیگنال پایین-فعال است. بنابراین، ارباب باید صفر منطقی را روی این سیگنال برای انتخاب برده‌ی مورد نظر ارسال کند. \واژه{spi} یک رابط تمام-دوطرفه است و هر دو ارباب و برده می‌توانند داده‌ها را به طور همزمان از طریق خطوط \واژه{mosi} و \واژه{miso} ارسال کنند. در طول ارتباط \واژه{spi}، داده‌ها به طور همزمان منتقل می‌شوند. لبه ساعت سریال، تغییر و نمونه گیری داده‌ها را همزمان می‌کند (شکل \رجوع{fig:spi}). رابط \واژه{spi} انعطاف پذیری را در اختیار کاربر قرار می‌دهد تا لبه بالا و پایین ساعت را برای نمونه برداری و یا تغییر داده انتخاب کند. معمولاً اطلاعات مربوط به تعیین تعداد بیت‌های داده منتقل شده با استفاده از رابط \واژه{spi}، در برگه داده دستگاه قید شده است\مرجع{spi-intro}.

در تراشه‌ی \چر{ESP8266}‌ یک واحد \واژه{qspi}‌ وجود دارد. \واژه{qspi} یک وسیله جانبی است که در اکثر میکروکنترلرهای مدرن یافت می‌شود. به طور خاص برای ارتباط با تراشه‌های فلش که از این رابط پشتیبانی می‌کنند طراحی شده است. این کاربرد به ویژه در برنامه‌هایی استفاده می‌شود که شامل داده‌های حجیم مانند محتوای چند رسانه ای است و حافظه روی تراشه کافی نیست. همچنین می‌توان از آن برای ذخیره سازی کد در بیرون از تراشه استفاده کرد و این قابلیت را دارد که سرعت حافظه خارجی را به سرعت حافظه داخلی از طریق برخی مکانیسم‌های نزدیک کند.

\واژه{qspi}
سریعتر از \واژه{spi} سنتی است زیرا از 4 خط داده (\چر{I0}، \چر{I1}، \چر{I2} و \چر{I3}) در مقایسه با 2 خط داده (\واژه{mosi} و \واژه{miso}) در \واژه{spi} سنتی استفاده می‌کند. حافظه فلش ارزان و بادوام است که آن را به گزینه ای جذاب برای برنامه‌های \واژه{emb} تبدیل می‌کند؛ اما طبیعتاً کند است و این باعث ایجاد تنگنا می‌شود و بر عملکرد برنامه‌های توکار تأثیر می‌گذارد. با وجود اینکه \واژه{spi} سریع است و سرعت آن می‌تواند تا 16 مگاهرتز افزایش یابد، دستگاه‌های فلش نمی‌توانند داده‌ها را با این سرعت از طریق یک خط داده واحد ارسال کنند تا سریعتر از حافظه روی تراشه کار کند.

قبل از ورود \واژه{qspi}، راه حل این مسئله استفاده از حافظه موازی بود که در آن 8، 16 یا 32 پین (بسته به محدوده آدرس) برای اتصال دستگاه حافظه خارجی با میکروکنترلر برای دستیابی به عملکرد سریع استفاده می‌شد. اما این رویکرد 2 نقص عمده داشت. اول این که طراحی \واژه{pcb} را پیچیده می‌کرد؛ و دوم ازدیاد پین‌های مورد نیاز بود و این بدان معناست که همه این پین‌ها که روی یک تراشه خاص تعداد ثابتی دارند دیگر نمی توانند برای دیگر کاربردها مورد استفاده قرار گیرند.

\شروع{شکل}[ht]
\تنظیم‌ازوسط
\درج‌تصویر[پهنا=\پهنای‌متن]{qspi}
\شرح{نحوه‌ی کار پروتکل \واژه‌متن{qspi} \مرجع{qspi}}
\برچسب{fig:qspi}
\پایان{شکل}

با توجه به همه این مشکلات، مهندسان نیاز به ارائه یک راه حل مناسب برای سریعتر ساختن حافظه‌ی فلش داشتند و راه حلی که آنها ارائه کردند این بود که پروتکل‌های \واژه{spi} را تغییر دهند تا از 2 خط داده دیگر استفاده کنند و هر 4 خط داده را دو طرفه کنند. برخلاف \واژه{spi} معمولی که از خطوط داده جداگانه برای ورودی و خروجی استفاده می‌کند، رابط \چر{Quad-\واژه{spi}} خطوط داده را در لحظه پیکربندی می‌کند تا در صورت نیاز به ارسال برخی اطلاعات به فلش مموری، آنها به عنوان خروجی عمل کنند؛ و در صورت نیاز به خواندن برخی از محتویات حافظه، به عنوان ورودی (شکل \رجوع{fig:qspi}).

برخی از مزایای حالت \چر{Quad-\واژه{spi}} شامل موارد زیر است:

\شروع{فقرات}

\فقره تعداد پین‌های پایین در مقایسه با حافظه‌های موازی، به این معنی است که \واژه{gpio} بیشتری برای سایر موارد استفاده می‌شود

\فقره طراحی \واژه{pcb} آسان تر است که نتیجه مستقیم تعداد کم پین است، که بیشتر به هزینه کلی پایین تر منجر می‌شود برخلاف گزینه‌های حافظه موازی، می‌توان از طراحی فشرده تری استفاده کرد

\فقره چندین دستگاه را می‌توان به یک رابط \واژه{qspi} متصل کرد و خطوط داده مشابه را می‌توان به چندین دستگاه متصل کرد. برای انتخاب یک تراشه خاص، می‌توان از پین انتخاب تراشه استفاده کرد. این یک مزیت است زیرا اگر از یک راه حل حافظه موازی استفاده کنیم، به مجموعه جداگانه ای از خطوط داده برای هر تراشه ای که با آن ارتباط برقرار می‌کنیم نیاز داریم.

\فقره به توان عملیاتی تقریباً 4 برابر \واژه{spi} معمولی می‌توان دست یافت

\فقره از اجرای کد در محل از طریق \واژه{xip} پشتیبانی می‌کند، که می‌تواند به افزایش حافظه کد و کارایی سیستم‌های پیچیده کمک کند. این ویژگی به میکروکنترلر اجازه می‌دهد کد را مستقیماً از روی فلش مموری خارجی بدون کپی اولیه اجرا کند که باعث می‌شود کد سریعتر و کارآمدتر اجرا شود.

\پایان{فقرات}

وقتی اندازه کد برای ذخیره در حافظه داخلی تراشه بسیار بزرگ می‌شود، ما معمولاً به سراغ حافظه خارجی می‌رویم، اما مشکل حافظه خارجی این بود که دسترسی به آن بسیار کند بود. اما با استفاده از حالت \چر{Quad-\واژه{spi}} و مکانیزم پیش واکشی، سرعت بازیابی اطلاعات دستگاه‌های فلش خارجی را می‌توان با حافظه داخلی تراشه‌ها مقایسه کرد و از این رو می‌توان نه تنها برخی پایگاه‌های داده و چند رسانه ای را ذخیره کرد، بلکه می‌توان از آنها برای اجرای کد نیز استفاده کرد\مرجع{qspi}.

در تراشه‌ی \چر{ESP8266}‌ یک واحد \واژه{qspi}‌ وجود دارد که از آن برای دسترسی به اطلاعات حافظه‌ی فلش خارجی و بارگذاری برنامه و نوشتن و ذخیره و اصلاح داده‌ی بر روی آن استفاده می‌شود. چگونگی بارگذاری و اجرای برنامه در این تراشه در بخشی مجزا به تفضیل بررسی خواهد شد.

علاوه بر واحد \واژه{qspi}، یک واحد \واژه{spi} معمولی نیز برای کاربرد دلخواه کاربر وجود دارد و قابل استفاده برای ادوات خارجی سازگار با این پروتکل است\مرجع{esp-ref}.

\زیرزیرقسمت{ورودی و خروجی همه منظوره}

ورودی/خروجی همه منظوره (\واژه{gpio}) رابطی است که در اکثر میکروکنترلرهای مدرن موجود است و راهی برای دسترسی به امکانات داخلی ابزارها ارائه می‌دهد. به طور کلی چندین پین \واژه{gpio} در یک میکروکنترلر برای استفاده از چندین کاربرد همزمان وجود دارد. پین‌ها را می‌توان به عنوان ورودی برنامه ریزی کرد، به طوری که داده‌های برخی از منابع خارجی در سیستم وارد می‌شوند تا در زمان و مکان دلخواه پردازش شوند. اعمال خروجی را می‌توان در \واژه{gpio}‌ها انجام داد، به صورتی که اطلاعات مورد نظر به طور مؤثر به دستگاه‌های خارجی منتقل شود؛ این یک مکانیسم ساده برای برنامه ریزی و ارسال داده‌ها بسته به خواسته‌های کاربر از طریق یک رابط تک پورت فراهم می‌کند. پین ها
معمولاً در گروه‌های ۸ تایی قرار می‌گیرند که در آن سیگنال‌ها می‌توانند به دستگاه‌های دیگر ارسال و از آن‌ها دریافت شوند. در بسیاری از برنامه ها، \واژه{gpio}‌ها را می‌توان به عنوان خطوط وقفه پیکربندی کرد
تا سیگنال‌های ورودی فوراً پردازش شوند. در بسیاری از طرح‌های جدیدتر، آنها
همچنین توانایی کنترل و استفاده از ادوات دسترسی مستقیم حافظه (\واژه{dma}) را برای انتقال بلوک‌های داده به طرز کارآمدتر و بهینه‌تری دارند. اساساً همه پورت‌ها می‌توانند متناسب با اهداف خاص طراحی به کار برده شوند و قابلیت استفاده مجدد در کاربردها را فراهم کنند. هر \واژه{gpio} باید بتواند در حالت ورودی یا خروجی برای هر پین روی تراشه تعریف شود\مرجع{gpio}.

این تراشه دارای ۱۷ پین قابل استفاده به صورت ورودی و خروجی همه منظوره است؛ گرچه برخی این پین‌ها با دیگر ادواتی از جمله رابط سریال و \واژه{spi}، مبدل قیاسی به رقمی و رابط I2S مشترک بوده و استفاده‌ی همزمان از حالت \واژه{gpio} ‌یا رابط دیگر در آن واحد ممکن نیست\مرجع{esp-ref}.

\زیرزیرقسمت{\واژه‌متن{uart-def}}

\واژه{uart}
یکی از ساده‌ترین و قدیمی‌ترین اشکال ارتباطات دیجیتالی دستگاه به دستگاه است. می‌توانید دستگاه‌های \واژه{uart} را به عنوان بخشی از مدارهای مجتمع یا به عنوان اجزای جداگانه پیدا کنید. \واژه{uart}‌ها با استفاده از یک جفت سیم و یک گره مشترک بین دو گره جداگانه ارتباط برقرار می‌کنند.

به عنوان یک راه حل «فراگیر»، می‌توانیم \واژه{uart} را طوری تنظیم کنیم که با انواع مختلف پروتکل‌های سریال کار کند. \واژه{uart} در اوایل دهه 1970 با واحدهای تک تراشه سازگار شد و با WD1402A وسترن دیجیتال شروع شد.

در یک ارتباط \واژه{uart}:

\شروع{شمارش}
\فقره پین Tx (انتقال) یک تراشه مستقیماً به پین ​​Rx (دریافت) دیگری متصل می‌شود و بالعکس. به طور معمول، انتقال در 3.3 یا 5 ولت انجام می‌شود. \واژه{uart} یک پروتکل تک کاره و تک برده است که در آن یک دستگاه برای ارتباط تنها با یک جفت راه اندازی شده است.

\فقره داده‌ها به موازات دستگاه کنترل (به عنوان مثال، \چر{CPU}) به یک \واژه{uart} حرکت می‌کنند.

\فقره هنگام ارسال پین Tx، اولین \واژه{uart} این اطلاعات موازی را به سریال ترجمه می‌کند و به همتای دریافت کننده منتقل می‌کند.

\فقره \واژه{uart} دوم این داده‌ها را روی پین Rx خود دریافت کرده و مجدداً به موازات تبدیل می‌کند تا با دستگاه کنترل کننده خود ارتباط برقرار کند.

\پایان{شمارش}

\چر{\واژه{uart}}ها
داده‌ها را به صورت سری در یکی از این سه حالت انتقال می‌دهند:

\شروع{فقرات}
\فقره
دوطرفه کامل: ارتباط همزمان با و از طرف هر ارباب و برده
\فقره
 نیمه دوطرفه: داده‌ها در یک زمان در یک جهت جریان می‌یابند
\فقره
ساده: فقط ارتباط یک طرفه
\پایان{فقرات}

انتقال داده‌ها در قالب بسته‌های داده انجام می‌شود، با «بیت شروع» شروع می‌شود، جایی که ولتاژ خط معمولی-بالا به ولتاژ زمین کشیده می‌شود. پس از بیت شروع، پنج تا نه بیت داده در آنچه به عنوان قاب داده بسته شناخته می‌شود، منتقل می‌شود، و سپس یک بیت برابری اختیاری برای تأیید صحت انتقال داده‌ها ارسال می‌شود. در نهایت، یک یا چند بیت توقف به طوری که خط روی بالا تنظیم شده است، منتقل می‌شود. با این کار یک بسته به پایان می‌رسد.

به عنوان یک پروتکل ناهمزمان - که یعنی هیچ خط ساعتی سرعت انتقال داده را تنظیم نمی کند - کاربران باید هر دو دستگاه را طوری تنظیم کنند که با سرعت یکسان ارتباط برقرار کنند. این سرعت به عنوان نرخ باود شناخته می‌شود که بر حسب بیت در ثانیه یا \واژه{bps} بیان می‌شود. سرعت انتقال به طور چشمگیری متفاوت است، از تنظیم معمولی 9600 باود تا 115200 و فراتر از آن.

\واژه{uart}
در حالی که چیزی شبیه یک پروتکل «باستانی» است و فقط می‌تواند بین یک ارباب و برده واحد ارتباط برقرار کند، به خوبی شناخته شده است، راه اندازی آن آسان است و بسیار همه کاره است. به این ترتیب، هنگام کار با پروژه‌های میکروکنترلر، احتمالاً با این سیستم برخورد خواهید کرد. \چر{\واژه{uart}}ها ممکن است بخشی از سیستم‌هایی باشند که هر روز از آن استفاده می‌کنید\مرجع{uart}.

تراشه‌ی \چر{ESP8266} دارای دو واحد \واژه{uart}‌ است. واحد اول از هر دو جهت فرستندگی و گیرندگی همزمان پشتیبانی می‌کند؛ ولی واحد دوم تنها می‌تواند به صورت فرستنده‌ی صرف استفاده شود\مرجع{esp-ref}. در این پروژه برای برقراری ارتباط با ریسه‌ی \چر{LED}ها و پیاده‌سازی پروتکل مبتنی بر روش مدولاسیون کد پالس (\واژه{pcm}) به روش هوشمندانه‌ای استفاده می‌شود تا میکروکنترلر بتواند طبق پروتکل با ریسه و در سرعت مناسب ارتباط برقرار کند. توضیحات مربوط به این روش در فصل مربوط به جزئیات پیاده‌سازی پروژه آورده شده است.

\زیرزیرقسمت{مبدل قیاسی به رقمی}

مبدلهای قیاسی به رقمی، که به اختصار \واژه{adc} نامیده می‌شوند، برای تبدیل سیگنالهای قیاسی (پیوسته، بی نهایت متغیر) به سیگنالهای رقمی (زمان گسسته، دامنه گسسته) کار می‌کنند. به عبارت کاربردی تر، ADC یک ورودی آنالوگ، مانند میکروفون که صدا را جمع آوری می‌کند، به سیگنال دیجیتال تبدیل می‌کند.

ADC
این تبدیل را به نوعی کوانتیزه انجام می‌دهد؛ یعنی مجموعه پیوسته مقادیر را به مجموعه کوچکتر (قابل شمارش) مقادیر، اغلب با گرد کردن، ترسیم می‌کند. در نتیجه، فرآیند آنالوگ به دیجیتال همیشه شامل مقدار مشخصی نویز یا خطا، هر چند کوچک است.

این تراشه دارای یک مبدل ۱۰ بیتی از نوع تقریب متوالی است\مرجع{esp-ref}. این نوع از مبدل‌ها به صورت بیت به بیت و با مقایسه‌های مکرر بین مقدار واقعی و مقدار تقریب زده شده، دقت مقدار اخیر را به صورت لگاریتمی (به تعداد بیت‌ها) افزایش می‌دهند\مرجع{adc}.
