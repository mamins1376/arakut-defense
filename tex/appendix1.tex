\فصل{کد منبع برنامه}
\برچسب{chap:code}

در این پیوست، کد منبع برنامه به همراه توضیحاتی از نحوه کارکرد آن آورده شده است.

\قسمت{فایل اصلی \واژه‌متن{firmware}}

در این برنامه از \واژه{pio} و کتابخانه \واژه{HAL} استفاده شده است. در ابتدا یک وب سرور راه اندازی می‌شود و ماژول کلاس مربوط به \واژه{ws}‌ آماده می‌گردد. سپس روتین‌های مدیریت درخواست‌های \واژه{http} تعریف می‌شوند و یک پروتکل ساده برای ارتباط با \واژه{wapp} که در اولین درخواست روی مرورگر کاربر بارگزاری می‌شود نیز تعریف می‌گردد.

در روتین \چر{setup}‌ کتابخانه‌ی مدیریت \واژه{wifi} آماده می‌شود. این کتابخانه امکان تعیین یک نقطه دسترسی جداگانه را می‌دهد که اطلاعات آن در حافظه دائمی ماژول برای اتصال خودکار در دفعات بعدی ذخیره می‌گردد.

\شروع{لاتین}\inputminted{c}{code/main.c}\پایان{لاتین}

\قسمت{فایل \واژه‌متن{wapp}}

این فایل به هنگام ساخت \واژه{firmware} ‌در دو مرحله فشرده می‌شود. ابتدا فضاهای خالی این فایل حذف شده و نام متغییرها با نامی کوتاه‌تر جایگزین می‌شود. در مرحله‌ی بعد با استفاده از الگوریتم \واژه{gz}‌فشرده سازی شده و مستقیما بر روی حافظه فلش ماژول ذخیره می‌گردد. به هنگام دریافت درخواست مرورگر از طرف کاربر، برنامه اصلی \واژه{firmware}‌ این داده‌ی تولید شده را مستقیماً به مرورگر می‌فرستد تا در آنجا از حالت فشرده خارج شود.

\شروع{لاتین}\inputminted{html}{code/public/index.html}\پایان{لاتین}

سپس برنامه روی مرورگر اجرا شده و از طریق \واژه{ws}‌ که روی ماژول آماده است تعین کردن سرعت و جهت موتورها را فراهم می‌سازد.

\شروع{لاتین}\inputminted{js}{code/public/main.js}\پایان{لاتین}


\قسمت{کد منبع}

کد زیر با استفاده از فریمورک میکروپایتون روی ماژول ESP‌ اجرا شده و به عنوان رابط بین برنامه مبتنی بر وب و میکروکنترلر فعالیت می‌کند.
\شروع{لاتین}\inputminted{python}{code/boot.py}\پایان{لاتین}

\شروع{لاتین}\inputminted{python}{code/server.py}\پایان{لاتین}



