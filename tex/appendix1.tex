\فصل{برنامه اولیه}
\برچسب{chap:code}

در این پیوست، کد منبع برنامه اولیه (قبل از استفاده از \چر{WLED}) به همراه توضیحاتی از نحوه کارکرد آن آورده شده است. کد منبع این برنامه روی سامانه اشتراک کد \واژه{gh} در دسترس است\مرجع{espws2811}.

\قسمت{فایل اصلی \واژه‌متن{firmware}}

در این برنامه از \واژه{pio} و کتابخانه \واژه{arduino} استفاده شده است. در ابتدا یک وب سرور راه اندازی می‌شود و ماژول کلاس مربوط به \واژه{ws}‌ آماده می‌گردد. سپس روتین‌های مدیریت درخواست‌های \واژه{http} تعریف می‌شوند و یک پروتکل ساده برای ارتباط با \واژه{wapp} که در اولین درخواست روی مرورگر کاربر بارگزاری می‌شود نیز تعریف می‌گردد.

در روتین \چر{setup}‌ ریسه‌ی دیودها راه‌اندازی اولیه می‌شود و سپس کتابخانه‌ی مدیریت \واژه{wifi} آماده می‌شود. این کتابخانه امکان تعیین یک نقطه دسترسی جداگانه را می‌دهد که اطلاعات آن در حافظه دائمی ماژول برای اتصال خودکار در دفعات بعدی ذخیره می‌گردد.

\شروع{لاتین}
\inputminted{C++}{espws2811/src/main.cpp}
\پایان{لاتین}

\قسمت{پروتکل \چر{WS2811}}

برای برقراری ارتباط با ریسه طبق پروتکل، از ادوات جانبی UART در حالت فرستنده معکوس شده استفاده می‌شود، بدین معنا که روی خروجی آن یک نفی کننده منطقی قرار گرفته است. سپس با جدولی که در خط ۱۷ کد مشاهده می‌شود، از هر دو بیت از ۲۴ بیت لازم برای ارسال یک رنگ یک بایت برای ارسال روی واحد سریال ایجاد می‌شود. اعداد ثبت شده در این جدول به نحوی تعیین شده اند تا با استفاده از بیت‌های واحد سریال، عملی شبیه \واژه{pcm} انجام شود که از لحاظ خطای زمانی، مطابق برگه اطلاعات پروتکل ریسه است (برای مشاهده مقدار خطای مجاز، \مرجع{ws2811:datasheet} را ببینید). بدین ترتیب، بدون استفاده از ورودی و خروجی همه منظوره و بدون استفاده از روش \واژه{bitbanging}، ریسه‌ی دیودها مدیریت می‌شود.

\واژه{buffer}هایی
که در خطوط ۹ و ۱۱ تعریف شده اند، به ترتیب رنگ ۲۴ بیتی و بایت‌های محاسبه شده را \واژه{cache}‌ می‌کنند. مزیت این کار در به وجود آمدن امکان تعیین رنگ تکی برای هر دیود در هر آدرسی است. بدون بافر رنگ مستقیم، برای هر بروزرسانی نیاز است که رنگ تمام دیودهای قبل از آن دوباره روی شبکه ارسال شوند. بافر پروتکل هم برای جلوگیری از محاسبه‌ی مجدد بایت‌های قبلی به کار می‌رود.

\شروع{لاتین}
\inputminted{C}{espws2811/src/ws2811.c}
\پایان{لاتین}

\قسمت{فایل \واژه‌متن{wapp}}

در این فایل \واژه{wapp} نوشته شده با استفاده از کتابخانه‌ \واژه{preact} قرار دارد. این کتابخانه برای طراحی برنامه‌های وب بسیار کم حجم به کار می‌رود\مرجع{preact}. این فایل به هنگام ساخت \واژه{firmware} ‌در دو مرحله فشرده می‌شود. ابتدا فضاهای خالی این فایل حذف شده و نام متغییرها با نامی کوتاه‌تر جایگزین می‌شود. در مرحله‌ی بعد با استفاده از الگوریتم \واژه{gz}‌فشرده سازی شده و مستقیما بر روی حافظه فلش ماژول ذخیره می‌گردد. به هنگام دریافت درخواست مرورگر از طرف کاربر، برنامه اصلی \واژه{firmware}‌ این داده‌ی تولید شده را مستقیماً به مرورگر می‌فرستد تا در آنجا از حالت فشرده خارج شود.

سپس برنامه روی مرورگر اجرا شده و از طریق \واژه{ws}‌ که روی ماژول آماده است تعین کردن رنگ ریسه را فراهم می‌سازد.

\شروع{لاتین}
\inputminted{jsx}{espws2811/web/index.jsx}
\پایان{لاتین}

