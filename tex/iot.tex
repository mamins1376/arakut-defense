\فصل{\واژه‌متن{iot}}
\برچسب{chap:iot}

\قسمت{\واژه‌متن{iot}}

\واژه{iot} سیستمی از ابزارهای کامپیوتری، مکانیکی، ماشین‌های دیجیتال، چیزها، حیوانات و افراد به هم متصل است که یک شناسه‌ی منحصر به فرد دارند و می‌توانند بدون وجود یک واسط انسانی (انسان به انسان و یا انسان به ماشین) با هم تبادل داده کنند.

یک «چیز» در \واژه{iot}/چیزها می‌تواند شخصی با یک ایمپلت در قلبش باشد، یک حیوان اهلی با ابزارهای اندازه‌گیری مشخصات بیولوژیکی باشد، یک خودرو که به تایرهای آن حسگر فشار متصل است تا کم‌بادی لاستیک‌ها را بسنجد، یا هر چیز دیگری که می‌توان به آن یک \واژه{ipaddr} اختصاص داد و می‌تواند روی شبکه داده منتقل کند\مرجع{blog:techtarget-iot}.

اتحادیه بین‌المللی مخابرات \واژه{iot} را «زیرساختی جهانی برای جامعه اطلاعاتی که بر اساس فناوری‌های ارتباطی و اطلاعاتی دارای قابلیت تعامل‌پذیری از قبل موجود و رو به رشد از طریق اتصال (فیزیکی و مجازی) اشیاء خدمات پیشرفته‌ای را ممکن می‌سازد» تعریف کرده‌است. طبق رهنمودهای اتحادیه بین‌المللی مخابرات «چیز» در عبارت \واژه{iot} یا به یک شی از جهان فیزیکی (اشیاءی فیزیکی) یا جهان اطلاعات (اشیاءی مجازی) اشاره دارد که قابلیت شناسایی شدن و یکپارچه گشتن با شبکه‌های ارتباطی را دارا است.
از لحاظ استانداردسازی‌های فنی، می‌توان به اینترنت اشیا به عنوان یک زیرساخت جهانی برای جامعه اطلاعات نگریست که می‌تواند با برقراری ارتباط (فیزیکی یا مجازی) میان چیزها، سبب ارائه خدمات پیشرفته شود.
در این حوزه، اشیا شامل موجودات دنیای فیزیکی (چیزهای قابل لمس) و موجودات دنیای اطلاعات (دنیای مجازی) هستند. به هر شئی می‌توان اطلاعاتی نسبت داد که می‌تواند ایستا و پویا باشد\مرجع{iot-overview}.

سازمان‌ها در برخی صنایع به صورت فزاینده‌ای از \واژه{iot} استفاده می‌کنند تا بهره‌وری را بالا ببرند، مشتریان را بهتر درک کنند تا به آن‌ها خدمات بهتری ارائه، تصمیم‌گیری خود را بهبود و ارزش کسب و کار خود را افزایش دهند\مرجع{blog:techtarget-iot}.

\زیرقسمت{مروری فنی بر اینترنت اشیا}

یک چیز فیزیکی ممکن است از طریق یک یا چند مورد مجازی در جهان اطلاعات نمایش داده شود (نگاشت)، اما یک چیز مجازی می‌تواند بدون هیچ چیز فیزیکی مرتبط نیز وجود داشته باشد.
دستگاه‌ها با دستگاه‌های دیگر ارتباط برقرار می‌کنند. آنها یا از طریق شبکه ارتباطی از طریق یک دروازه (مورد الف)، یا از طریق شبکه ارتباطی بدون دروازه (مورد ب)، یا مستقیماً بدون استفاده از شبکه ارتباطی (مورد پ) ارتباط برقرار می‌کنند. همچنین، ترکیب موارد الف و پ، و موارد ب و پ امکان پذیر است. به عنوان مثال، دستگاه‌ها می‌توانند با استفاده از ارتباط مستقیم از طریق یک شبکه محلی (یعنی یک شبکه که اتصال محلی میان دستگاه‌ها و میان دستگاه و دروازه را برقرار می‌کند، چون شبکه \چر{ad-hoc}) (مورد پ) و سپس با برقراری ارتباط توسط یک دروازه شبکه در سراسر شبکه ارتباط (مورد الف)، با سایر دستگاه‌ها در ارتباط باشند\مرجع{iot-overview}.

\زیرقسمت{نحوه کار \واژه‌متن{iot}}

یک اکوسیستم \واژه{iot}، متشکل از ابزارهای هوشمند مبتنی بر وب هستند که از سامانه‌های نهفته مانند پردازنده‌ها، حسگرها و سخت‌افزارهای ارتباطی استفاده می‌کنند تا داده‌هایی که از محیط خود جمع‌آوری نموده را ذخیره، ارسال و یا پردازش کنند. ابزارهای \واژه{iot} داده‌های جمع‌آوری شده از حسگرهای خود را از طریق یک درگاه \واژه{iot} یا یک ابزار واسطه‌ی دیگر که اطلاعات را به یک \واژه{cloud-computer} می‌فرستد تا در آنجا تحلیل شود به اشتراک می‌گذارند (شکل \رجوع{fig:iot2}).

\شروع{شکل}[ht]
\تنظیم‌ازوسط
\درج‌تصویر[پهنا=\پهنای‌متن]{iot2}
\شرح{طرحی از تنوع ابزارهای قابل استفاده در \واژه‌متن{iot}}
\برچسب{fig:iot2}
\پایان{شکل}

بعضاً این ابزارها اطلاعاتی که دریافت کرده‌اند را یه ابزارهای مربوط و مشابه خود ارسال می‌کنند تا به کمک یکدیگر روی آن عملی انجام دهند. ابزارها اکثر کار را بدون دخالت انسان انجام می‌دهند، گرچه افراد نیز می‌توانند با این ابزارها تعامل داشته باشند؛ به عنوان مثال آن‌ها را راه‌اندازی کنند، به آن‌ها فرمان بدهند و یا داده‌های آن‌ها را دریافت کنند.

نحوه‌ی اتصال، شبکه‌سازی و پروتکل‌های مخابراتی استفاده‌شده با این ابزارهای مبتنی بر وب بستگی زیادی به کاربرد خاصی که در بستر \واژه{iot} قرار است پیاده‌سازی شود دارد.

\واژه{iot}
می‌تواند از دانش \واژه{ai} و \واژه{ml} استفاده کند تا جمع‌آوری داده‌ها راحت‌تر و پویاتر گردد\مرجع{blog:techtarget-iot}.

\زیرقسمت{اهمیت \واژه‌متن{iot}}

این تکنولوژی به افراد کمک می‌کند تا زندگی هوشمندتری داشته و مؤثرتر کار کنند و همچنین کنترل بیشتری روی اطرافشان داشته باشند. برای هوشمندسازی خانه‌ها و اتوماسیون خانگی، استفاده از \واژه{iot} الزامی است. \واژه{iot} به کسب و کارها دید لحظه‌ای به این که سامانه‌هایشان دقیقاً در حال انجام چه کاری است می‌دهد که موجب می‌شود درک عمیقی از بازدهی ماشین‌ها و چرخه‌ی تأمین و عملیات لجستیک به دست آید.

از مزایای \واژه{iot}، خودکارسازی فرآیندها و کاهش هزینه‌ی کار برای شرکت‌ها است. با کاستن هدررفت و بهبود ارائه‌ی خدمات، ارسال کالاها را ارزانتر کرده و شفافیت بیشتری به مشتریان عرضه می‌دارد.

به همین صورت، \واژه{iot} یکی از مهم‌ترین تکنولوژی‌های روزمره است و گستردگی آن همان طور که کسب و کارها پتانسیل این روش را درک می‌کنند تا ارزش رقابتی آن‌ها افزایش یابد، بیشتر می‌شود\مرجع{blog:techtarget-iot}.

\زیرقسمت{مزایا و معایب \واژه‌متن{iot}}

برخی از مزایای اینترنت اشیا شامل موارد زیر است:
\شروع{فقرات}
\فقره امکان دسترسی به اطلاعات از هر نقطه در هر زمان در هر دستگاه،
\فقره بهبود ارتباط بین دستگاه‌های الکترونیکی متصل،
\فقره انتقال بسته‌های داده بر بستر شبکه متصل و ذخیره‌ی زمان و هزینه
\فقره اتوماسیون وظایف برای بهبود کیفیت خدمات یک تجارت و کاهش نیاز به مداخله انسانی.
\پایان{فقرات}

برخی از معایب اینترنت اشیا شامل موارد زیر است:
\شروع{فقرات}
\فقره با افزایش تعداد دستگاه‌های متصل و به اشتراک گذاری اطلاعات بیشتر بین دستگاه ها، احتمال سرقت اطلاعات محرمانه توسط یک هکرها نیز افزایش می‌یابد.
\فقره ممکن است شرکت‌ها در نهایت مجبور به مواجهه با تعداد زیادی - شاید میلیون‌ها - دستگاه اینترنت اشیا شوند و جمع آوری و مدیریت داده‌ها از همه این دستگاه‌ها چالش برانگیز شود.
\فقره اگر اشکالی در سیستم وجود داشته باشد، احتمالاً هر دستگاه متصل خراب می‌شود.
\فقره از آنجا که هیچ استاندارد بین المللی سازگاری برای \واژه{iot} وجود ندارد، برقراری ارتباط بین دستگاه‌های سازندگان مختلف دشوار است\مرجع{blog:techtarget-iot}.
\پایان{فقرات}

\زیرزیرقسمت{مزایای \واژه{iot} برای سازمان‌ها}

\واژه{iot} چندین مزیت برای سازمان‌ها به ارمغان می‌آورد. برخی از این مزایا برای صنایع بخصوصی هستند و برخی دیگر به صنایع بسیاری کمک می‌کنند. برخی از این مزایا عبارت‌اند از:
\شروع{فقرات}
\فقره روند کلی فرآیندهای مختلف را نظارت کنند،
\فقره تجربه‌ی مشتریان را بهبود بخشند،
\فقره صرفه‌ی اقتصادی و زمانی را افزایش دهند،
\فقره بازدهی کارکنان را افزایش دهند،
\فقره مدل‌های کسب و کار بهتری اتخاذ کنند،
\فقره تصمیم‌های مدیریتی بهتری با داده‌های به دست آمده بگیرند و
\فقره درآمد بیشتری کسب کنند.
\پایان{فقرات}

\واژه{iot} شرکت‌ها را تشویق می‌کند تا نحوه‌ی برخورد با کسب و کارشان را بازنگری کرده و ابزاری را در اختیار آن‌ها قرار می‌دهد تا راهکارهای مدیریتی خود را بهتر کنند.

به طور کلی، در صنایع و سازمان‌های ارائه‌دهنده‌ی خدمات ساخت و تولید، حمل و نقل و صنایع تأسیساتی که از حسگرها و دیگر ابزار از این قبیل استفاده‌ی زیادی دارند، \واژه{iot} وافرترین کاربرد را دارد؛ گرچه در صنایع کشاورزی، زیرساخت و اتوماسیون خانگی نیز کاربردهایی مشاهده می‌شود.

\واژه{iot} می‌تواند با سهولت کار کشاورزان در بخش کشاورزی سودمند باشد. حسگرها می‌توانند اطلاعات مربوط به بارندگی، رطوبت، دما و محتوای خاک و سایر عوامل را که به خودکارسازی تکنیک‌های کشاورزی کمک می‌کند، جمع آوری کنند.

توانایی نظارت بر عملیات زیرساخت‌ها نیز عاملی است که \واژه{iot} می‌تواند به آن کمک کند. به عنوان مثال، می‌توان از حسگرها برای نظارت بر رویدادها یا تغییرات در \glspl{struct-building}، پل‌ها و سایر زیرساخت‌ها استفاده کرد. این امر مزایایی مانند صرفه‌جویی در هزینه، زمان، تغییرات گردش کار در کیفیت زندگی و گردش کار بدون نیاز به کاغذ را به همراه دارد.

یک تجارت مربوط به اتوماسیون خانگی می‌تواند از اینترنت اشیا برای نظارت و ایجاد تغییرات در سیستم‌های مکانیکی و الکتریکی در یک ساختمان استفاده کند. در مقیاسی وسیع‌تر، شهرهای هوشمند می‌توانند به شهروندان در کاهش ضایعات و مصرف انرژی کمک کنند.

اینترنت اشیا همه صنایع را تحت تأثیر قرار می‌دهد، از جمله مشاغل مربوط به مراقبت‌های بهداشتی، مالی، خرده فروشی و تولید\مرجع{blog:techtarget-iot}.

\زیرقسمت{خصوصیت‌های بنیادی}
از جمله خصوصیات بنیادی اینترنت اشیا، می‌توان به موارد زیر اشاره کرد:

\شروع{فقرات}
\فقره \متن‌سیاه{ارتباطات متقابل}:
در حوزه اینترنت اشیا، هر چیزی می‌تواند با زیرساخت‌های اطلاعاتی و ارتباطی جهانی ارتباط داشته باشد.
خدمات مرتبط با اشیاء: اینترنت اشیا قادر به ارائه خدمات مرتبط با چیزها در محدوده موارد، مانند حفاظت از حریم خصوصی و سازگاری معنایی بین اشیاء فیزیکی و اشیاء مجازی مرتبط با آنها است. به منظور ارائه خدمات مرتبط با چیزها در محدودیت‌های اشیاء، هم فناوری‌های دنیای فیزیکی و هم اطلاعات جهان تغییر خواهند کرد.

\فقره \متن‌سیاه{ناهمگنی}:
دستگاه‌های موجود در اینترنت اشیا بر اساس پلتفرم‌ها و شبکه‌های سخت افزاری مختلف ناهمگن هستند. آنها می‌توانند از طریق شبکه‌های مختلف با سایر دستگاه‌ها یا سیستم عامل‌های خدمات ارتباط برقرار کنند.

\فقره \متن‌سیاه{تغییرات پویا}:
 وضعیت دستگاه‌ها به طور پویا تغییر می‌کند، به عنوان مثال، خوابیدن و بیدار شدن، اتصال و/یا قطع شدن و همچنین زمینه دستگاه‌ها از جمله مکان و سرعت. علاوه بر این، تعداد دستگاه‌ها می‌تواند به صورت پویا تغییر کند.

\فقره \متن‌سیاه{مقیاس عظیم}:
تعداد دستگاه‌هایی که باید مدیریت شوند و با یکدیگر ارتباط دارند حداقل یک مرتبه بزرگتر از دستگاههای متصل به اینترنت فعلی خواهد بود. نسبت ارتباطات ایجاد شده توسط دستگاه‌ها در مقایسه با ارتباطات ایجاد شده توسط انسان به طور قابل ملاحظه ای به سمت دستگاه ایجاد می‌شود.
ارتباط حتی مهمتر مدیریت داده‌های تولید شده و تفسیر آنها برای اهداف کاربردی خواهد بود. این امر به معناشناسی داده‌ها و همچنین مدیریت کارآمد داده‌ها مربوط می‌شود\مرجع{iot-overview}.

\پایان{فقرات}

\زیرقسمت{الزامات سطح بالا}

موارد زیر برخی از الزامات سطح بالای مربوط به اینترنت اشیا هستند:

\شروع{فقرات}

\فقره \متن‌سیاه{اتصال مبتنی بر شناسایی}:
اینترنت اشیا باید تضمین کند که ارتباط بین یک چیز و اینترنت اشیا بر اساس شناسه آن چیز برقرار شود. این همچنین شامل این است که شناسه‌های احتمالاً ناهمگن از چیزهای مختلف ،به صورت یکپارچه پردازش می‌شوند.

\فقره \متن‌سیاه{قابلیت همکاری}:
همکاری بین سیستمهای ناهمگن و توزیع شده برای ارائه و مصرف انواع اطلاعات و خدمات باید تضمین شود.

\فقره \متن‌سیاه{شبکه‌های خودمختار}:
شبکه‌های خودمختار (شامل خودمدیریت، خودپیکربندی، درمان خود، تکنیک‌ها و/یا مکانیسم‌های خود بهینه سازی و محافظت از خود) برای عملکرد در شبکه‌های کنترل اینترنت اشیا، به منظور سازگاری با دامنه‌های مختلف برنامه، محیط‌های ارتباطی مختلف و تعداد و انواع دستگاه ها.

\فقره \متن‌سیاه{ارائه خدمات خودمختار}:
خدمات باید بتوانند با ضبط، انتقال و پردازش خودکار داده‌های موارد بر اساس قوانین پیکربندی شده توسط اپراتورها یا سفارشی توسط مشترکین، ارائه شوند. خدمات خودمختار ممکن است به تکنیک‌های تلفیق خودکار داده‌ها و داده کاوی بستگی داشته باشد.

\فقره \متن‌سیاه{قابلیت‌های مبتنی بر مکان}:
قابلیت‌های مبتنی بر مکان باید در اینترنت اشیا پشتیبانی شوند. ارتباطات و خدمات مربوط به چیزی بستگی به اطلاعات موقعیت مکانی اشیاء و/یا کاربران دارد. برای درک و ردیابی خودکار اطلاعات محل مورد نیاز است.
ارتباطات و خدمات مبتنی بر مکان ممکن است توسط قوانین و مقررات محدود شده باشد و باید با الزامات امنیتی مطابقت داشته باشد.

\فقره \متن‌سیاه{امنیت}:
در اینترنت اشیا، هر «چیزی» به هم متصل است که منجر به تهدیدهای امنیتی قابل توجهی می‌شود، مانند تهدید به محرمانه بودن، اصالت و یکپارچگی داده‌ها و خدمات.
یک مثال مهم از الزامات امنیتی نیاز به ادغام سیاست‌ها و تکنیک‌های مختلف امنیتی مربوط به انواع دستگاه‌ها و شبکه‌های کاربر در اینترنت اشیا است.

\فقره \متن‌سیاه{حفاظت از حریم خصوصی}:
حفاظت از حریم خصوصی باید در اینترنت اشیا پشتیبانی شود. بسیاری از چیزها صاحبان و کاربران خود را دارند. داده‌های حساس اشیاء ممکن است حاوی اطلاعات خصوصی مربوط به مالکان یا کاربران آنها باشد. اینترنت اشیا باید در حین انتقال داده ها، تجمیع، ذخیره سازی، استخراج و پردازش از حفاظت از حریم خصوصی پشتیبانی کند. حفاظت از حریم خصوصی نباید مانعی برای احراز هویت منبع داده ایجاد کند.

\فقره \متن‌سیاه{خدمات مربوط به بدن انسان با کیفیت بالا و بسیار ایمن}:
خدمات مربوط به بدن انسان با کیفیت بالا و بسیار ایمن باید در اینترنت اشیا پشتیبانی شود. کشورهای مختلف قوانین و مقررات متفاوتی در مورد این خدمات دارند (خدمات مربوط به بدن انسان به خدمات ارائه شده توسط ضبط، انتقال و پردازش داده‌های مربوط به ویژگی‌های ایستا و رفتار پویا با یا بدون دخالت انسان ارائه می‌شود).

\فقره \متن‌سیاه{مدیریت}:
برای اطمینان از عملکرد عادی شبکه، باید قابلیت مدیریت در \واژه{iot} پشتیبانی شود. برنامه‌های کاربردی اینترنت اشیا معمولاً بدون مشارکت افراد به طور خودکار کار می‌کنند، اما کل فرایند عملکرد آنها باید توسط طرفهای مربوطه قابل مدیریت باشد\مرجع{iot-overview}.

\پایان{فقرات}

\زیرقسمت{لایه شبکه}

شامل دو نوع قابلیت زیر است:

\شروع{فقرات}

\فقره \متن‌سیاه{قابلیت‌های شبکه}:
ارائه عملکردهای کنترل مربوط به اتصال شبکه، مانند عملکردهای کنترل منابع دسترسی و حمل و نقل، مدیریت تحرک یا احراز هویت، مجوز و حسابداری (خلاصه شده به صورت \چر{AAA}).

\فقره \متن‌سیاه{قابلیت‌های حمل و نقل}:
تمرکز بر ایجاد ارتباط برای انتقال خدمات اینترنت اشیا و اطلاعات داده‌های کاربردی خاص، و همچنین انتقال اطلاعات کنترل و مدیریت مربوط به اینترنت اشیا.

\پایان{فقرات}

\زیرقسمت{لایه دستگاه}

از نظر منطقی قابلیت‌های لایه دستگاه را می‌توان به دو نوع قابلیت طبقه بندی کرد.

\زیرزیرقسمت{قابلیت‌های دستگاه}

 قابلیت‌های دستگاه شامل موارد زیر است:
 
\شروع{فقرات}

\فقره \متن‌سیاه{تعامل مستقیم با شبکه ارتباطی}:
دستگاه‌ها قادر به جمع آوری و بارگذاری مستقیم اطلاعات (یعنی بدون استفاده از قابلیت‌های دروازه) در شبکه ارتباطی هستند و می‌توانند مستقیماً اطلاعات (مثل فرمان‌ها) را از شبکه ارتباطی دریافت کنند.

\فقره \متن‌سیاه{تعامل غیرمستقیم با شبکه ارتباطی}:
دستگاهها قادرند اطلاعات را به طور غیرمستقیم جمع آوری کرده و در شبکه ارتباطی بارگذاری کنند، یعنی از طریق قابلیتهای دروازه. از سوی دیگر، دستگاه‌ها می‌توانند به طور غیر مستقیم اطلاعات (مثل فرمان‌ها) را از شبکه ارتباطی دریافت کنند.

\فقره \متن‌سیاه{شبکه‌های موقت}:
دستگاه‌ها ممکن است بتوانند شبکه‌هایی را به صورت موقت در برخی از سناریوها ایجاد کنند که به افزایش مقیاس پذیری و استقرار سریع نیاز دارند.

\فقره \متن‌سیاه{خوابیدن و بیدار شدن}:
ظرفیت‌های دستگاه می‌توانند از مکانیسم‌های «خواب» و «بیداری» برای صرفه جویی در انرژی پشتیبانی کند.

\متن‌سیاه{توجه}
- پشتیبانی از هر دو قابلیت تعامل مستقیم با شبکه ارتباطی و تعامل غیر مستقیم با شبکه ارتباطی اجباری نیست\مرجع{iot-overview}.

\پایان{فقرات}

\زیرزیرقسمت{قابلیت‌های دروازه}

قابلیت‌های دروازه شامل موارد زیر است:

\شروع{فقرات}

\فقره \متن‌سیاه{پشتیبانی از چندین رابط}:
قابلیت‌های دروازه در لایه دستگاه، از طریق دستگاه‌هایی متصل می‌شوند که از طریق انواع مختلف فناوری‌های سیمی یا بی سیم متصل می‌شوند، همچون گذرگاه شبکه کنترل کننده (\چر{CAN})،\واژه{zigbee}،\واژه{bt} یا \واژه{wifi}. در لایه شبکه، قابلیت‌های دروازه ممکن است از طریق فن آوری‌های مختلفی چون شبکه تلفن عمومی سوئیچ شده (\چر{PSTN})، نسل دوم یا نسل سوم (\چر{2G} یا \چر{3G})، شبکه‌های تکاملی طولانی مدت (\چر{LTE})، اترنت یا خطوط مشترکین دیجیتالی (\چر{DSL}) ارتباط برقرار کنند.

\فقره \متن‌سیاه{تبدیل پروتکل}:
دو موقعیت وجود دارد که در آن به قابلیت‌های دروازه نیاز است. یکی از مواردی است که در ارتباطات در لایه دستگاه از پروتکل‌های لایه دستگاه مختلف استفاده می‌شود، مثلا پروتکل‌های فناوری زیگ‌بی و پروتکل‌های فناوری\واژه{bt} .حالت دیگر زمانی است که ارتباطات شامل لایه دستگاه و لایه شبکه از پروتکل‌های متفاوتی استفاده می‌کند؛ به عنوان مثال در پروتکل فناوری زیگ‌بی در لایه دستگاه و پروتکل فناوری \چر{3G} در لایه شبکه\مرجع{iot-overview}.

\پایان{فقرات}

\قسمت{\واژه‌متن{iot} در اتوماسیون خانگی}

اتوماسیون خانگی با استفاده از اینترنت اشیا مدرن ترین شیوه زندگی و راحتی برای شهروندان شهرهای هوشمند است. کسی که از روش‌های اتوماسیون خانگی در هنگام ساخت خانه استفاده می‌کند، می‌تواند مصرف انرژی را کنترل، نظارت و تنظیم کند. این فناوری هوشمند با استفاده از دستگاه‌هایی مانند یک سیستم گرمایش هوشمند، بهینه‌سازی منابع و انرژی را تسهیل می‌کند.

\شروع{شکل}[ht]
\تنظیم‌ازوسط
\درج‌تصویر[پهنا=0.6\پهنای‌متن]{smart-home}
\شرح{نمایی از بخش‌های مختلفی از یک خانه قابل کنترل با برنامه کاربردی مبتنی بر \واژه‌متن{iot}\مرجع{source:homify}.}
\برچسب{fig:smart-home}
\پایان{شکل}

چون از این ابزار برای کنترل، نظارت و تبدیل اطلاعات به دستگاه‌های دیگر بدون ارتباط فرد با فرد و یا فرد با کامپیوتر از طریق اینترنت استفاده می‌شود، می‌توان اقدامات خاصی را به طور خودکار در هر زمان که شرایط خاصی رخ می‌دهد فعال کرد.

به عنوان مثال، چراغ اتاق خواب به طور خودکار هنگام ورود شخص روشن می‌شود. چراغ‌های ناحیه خاصی در خانه را می‌توان طوری برنامه‌ریزی کرد که به طور خودکار روشن/خاموش شود. این کار تلاش‌های دستی را کاهش می‌دهد و برنامه‌های تلفن همراه اتوماسیون خانگی اینترنت اشیا می‌توانند زحمت آن را کمتر کند.

خانه مدرن از طریق اینترنت خودکار شده است و لوازم خانگی شما هر روز کنترل می‌شود. دستورات کاربر از طریق اینترنت توسط \واژه{wifi} دریافت می‌شود. میکروکنترلر داخل مودم دارای رابطی با شبکه خانه است و وضعیت سیستم از طریق صفحه نمایش همراه با داده‌های سیستم نمایش داده می‌شود. این یک سیستم اتوماسیون خانگی مبتنی بر اینترنت اشیا برای مدیریت کلیه لوازم خانگی شما است.

در اینجا برخی از کاربردهایی که اتوماسیون \واژه{iot} ممکن می‌کند آورده شده است:

\شروع{فقرات}
\فقره چراغ‌های هوشمند

چراغ‌های اتوماتیک می‌توانند روشن/خاموش بودن و میزان روشنایی را کنترل کنند. علاوه بر این، می‌تواتند به اقدامات مختلفی حساس نشان دهند؛ به عنوان مثال اگر شما تلویزیون تماشا م‌ کنید و فرمان می‌دهید که به تاریکی احتیاج دارید، چراغ‌ها خاموش می‌شوند و تجربه زیبایی را در محیط اطراف خود به شما می‌دهند؛ یا وقتی صبح زود بیدار می‌شوید چراغ‌ها به طور خودکار و سر ساعت و یا با حرکت فیزیکی شما روشن شده و با طلوع آفتاب و افزایش نور محیطی خاموش می‌شوند.

\فقره درب‌های هوشمند

درهای خودکار با استفاده از برنامه‌های \واژه{iot} کار می‌کنند. با استفاده از حسگرهای تلفن همراه خود و برنامه تنظیم شده برای ورودی خانه، می‌توانید درب پارکینگ و درب امنیتی را با استفاده از یک دستگاه هوشمند و برنامه تشخیص چهره به یک درب هوشمند تبدیل کنید.

درب هوشمند با استفاده از اینترنت اشیا می‌تواند ویژگی‌های اضافی را برای همگام‌سازی و خاموش کردن چراغ‌های خانه اضافه کند؛ مثلاً وقتی شخصی وارد خانه می‌شود و از آن خارج می‌شود؛ یا یک دروازه هوشمند از پارکینگ خودرو، فردی را که وارد خانه یا آسانسور می‌شود را حس می‌کند و کاربر را مطلع سازد تا پاسخ دهد یا به طور خودکار درب پارکینگ را باز و بسته می‌کند.

\فقره لوازم خانگی هوشمند

با استفاده از لوازم خانگی هوشمند که بسیار برنامه‌ریزی شده و از طریق اینترنت و برنامه‌های کاربردی مورد نیاز کار می‌کنند، می‌توان زندگی را ساده کرد.

مایکروویو می‌تواند برای گرم کردن غذا در درجه خاصی و در زمان خاصی عمل کند؛ هشدارهای هوشمند می‌توانند برای بیدار شدن تنظیم شوند؛ وسیله نقلیه با دقت بهتری نقص قطعات را تشخیص می‌دهد و الزامات سرویس را به شما اطلاع می‌دهد.

\فقره فعالیتهای معمول و روزمره

هشدارها و اعلان‌ها می‌توانند به کاربر کمک کنند تا کارهای روزمره خود را با دقت مدیریت کند. یک کاربر یک اعلان نامه پست الکترونیکی برای آنچه باید در یک زمان خاص بگیرد یا بفرستد دریافت می‌کند. همچنین، یک کاربر از لوازم خانگی یا هر چیزی که در ارتباط با برنامه اینترنت اشیا باشد یادآوری را دریافت می‌کند. به عنوان مثال هشدار متنی خودکار برای نشتی لوله در حمام. هنگامی که کاربر به انیار یا جایی که ابزارها در آن قرار دارند پا می‌گذارد، یک یادآور متنی دریافت می‌کند. اگر فر را برای پیش گرم کردن روشن کرده‌اید، وقتی فر با دمای ثابت گرم می‌شود به اطلاع شما خواهد رسید.

\فقره بهبود امنیت خانه

امنیت خانه برای همه مهمترین است. برنامه‌ی مخصوص شرایط امنیتی خانه حتی اگر خارج از خانه، شهر یا کشور هستید، خانه شما را نظارت و آن را ایمن می‌کند.

کاربر می‌تواند از راه دور با استفاده از تلفن هوشمند از هر نقطه در جهان به تصویر ویدئویی امنیتی دسترسی داشته باشد. یک دوربین امنیتی خانه نسبت به حرکت واکنش نشان می‌دهد و تشخیص می‌دهد که دسترسی غیرمجاز از طرف خارجی را باید با پیامک به شما اطلاع دهد.

کاربر می‌تواند چراغ‌های خانه را طوری تنظیم کند که در صورت نقض امنیت، با حداکثر نور روشن شود. اگر صاحب خانه از خانه خارج شود و قفل در اصلی را فراموش کرده باشد، قفل اتوماتیک بعد از سه دقیقه فعال خواهد شد\مرجع{blog:technostacks-iot-home}.
\پایان{فقرات}

\قسمت{سیستم‌های کنترلی مبتنی بر \واژه‌متن{iot}}

در صورتی که اگر از \واژه{iot} استفاده شود این نیاز مرتفع شده و نصب دستگاه‌های کنترلی جدید در مناطقی که حتی سابقاً امکان اضافه کردن سیم‌کشی جدید و یا اعمال اصلاحات در سیم‌کشی قبلی وجود نداشته است ممکن می‌شود. همچنین، هر دستگاه می‌تواند به صورت یک تکرار کننده عمل کرده و برد موثر برای کنترل ادوات کنترلی نیز بدون استفاده از فرستنده‌های پر توان‌تر که بعضاً ممکن است به علل فنی و یا اقتصادی ممکن نباشد افزایش یابد. دستگاه‌هایی که با فرستادن امواج رادیویی کار می‌کنند نیاز به اخذ مجوز از سازمان کنترل‌کننده در بدنه مدیریتی هر منطقه یا کشور دارد؛ که در ایران این وظیفه به عهده‌ی سازمان تنظیم مقررات و ارتباطات رادیویی است. جهت اعطای مجوز مربوط به دستگاه نیاز است که شرایط و ضوابطی توسط سازنده رعایت شود و یکی از آن‌ها محدودیت حداکثر توان رادیویی تشعشع شده از دستگاه است. این ضابطه موجب می‌شود که برد قابل تحقق برای هر پیوند رادیویی دارای حد بالای مشخصی باشد که امکان عبور از آن بدون تخطی از قوانین وضع شده وجود نخواهد داشت و عملاً ممکن نیست تا با استفاده از فرستنده‌ی قوی‌تر بتوان برد مورد نیاز را تأمین کرد.
