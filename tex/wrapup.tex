\فصل{نتیجه‌گیری}
\برچسب{chap:wrapup}

برای ساخت یک سیستم قابل‌کنترل موتوری با تکنولوژی‌های مرتبط با \واژه{iot} و استفاده از مزایایی که به همراه دارد از جمله کنترل سیستم از راه دور و بر بستر شبکه‌ی جهانی اینترنت، به ۴ بخش نیاز است که متشکل از درایور موتور مناسب، کنترلر سازگار، منبع تغذیه‌ی کارآمد، ابزاری برای تعامل با کنترلر و در صورت نیاز به گسترش برد اتصال، ادوات شبکه‌ی \واژه{wifi} است.

با وجود نرم‌افزارهای متن‌باز در این حوزه و امکانات و قابلیت‌های آن‌ها، نیازی به دانش برنامه‌نویسی برای ساخت یک کنترلر با کیفیت رفع شده است. هر فرد علاقه‌مندی می‌تواند با کمی اطلاعات در حوزه‌ی برق و توانایی برقراری اتصالات مناسب،‌ یک کنترلر ارزان قیمت برای خود بسازد.

