\فصل{نتیجه‌گیری}
\برچسب{chap:wrapup}

طراحی سیستم‌های نورپردازی پیشرفته بیش از پیش در دسترس عموم قرار گرفته است و میان کاربران خانگی نیز جای خود را پیدا کرده است. طراحان فضای داخلی منازل مسکونی می‌توانند با استفاده‌ی مناسب از این ابزار جلوه‌ی جدیدی به ظاهر کار خود بدهند و پویایی دلخواه خود را در آن ایجاد کنند؛ مانند نور محیطی ثابت و یا نورپردازی داخلی مناسب پخش موسیقی یا فیلم و یا مناسبت‌ها و مراسم خاص. همچنین، با نوآوری‌های انجام شده و به وجود آمدن امکان هماهنگ‌سازی با اجرای هنرمندان و تدوین برنامه‌ی نورپردازی، طراحان صحنه نیز به این ابزار به راحتی دسترسی خواهند داشت و در اجراهای هنری (مانند تئاتر و سینما و کنسرت‌های موسیقی) و اجراهای ورزشی (مانند ورزش‌های نمایشی گروهی و برنامه‌های تلوزیونی) می‌توانند به عمق و کیفیت برنامه‌ی خود بیافزایند.

برای ساخت یک سیستم نورپردازی قابل‌کنترل با تکنولوژی‌های مرتبط با \واژه{iot} و استفاده از مزایایی که به همراه دارد از جمله کنترل سیستم از راه دور و بر بستر شبکه‌ی جهانی اینترنت، به ۴ بخش نیاز است که متشکل از خروجی نوری مناسب، کنترلر سازگار، منبع تغذیه‌ی کارآمد، ابزاری برای تعامل با کنترلر و در صورت نیاز به گسترش برد اتصال، ادوات شبکه‌ی \واژه{wifi} است.

با وجود نرم‌افزارهای متن‌باز در این حوزه و امکانات و قابلیت‌های آن‌ها، نیازی به دانش برنامه‌نویسی برای ساخت یک کنترلر با کیفیت رفع شده است. هر فرد علاقه‌مندی می‌تواند با کمی اطلاعات در حوزه‌ی برق و توانایی برقراری اتصالات مناسب،‌ یک کنترلر ارزان قیمت برای خود بسازد و از آن در داخل خانه، محل‌کار، صحنه‌های هنری تئاتر یا موسیقی، اجراهای ورزشی و یا در خارج از یک ساختمان و به عنوان نمای بیرونی، در فضای باز و پارک‌ها و مناطقی که امکان سیم‌کشی برای هدایت سیستم‌های نوری به سادگی میسر نیست استفاده کند.

نرم‌افزارهای موجود امکان هماهنگ‌سازی و پخش جلوه‌های نوری هماهنگ با موسیقی را برای کاربر فراهم نموده و او را از تهیه‌ی برنامه‌ی نورپردازی هماهنگ با موسیقی به روش‌های دستی و وقت‌گیر بی نیاز می‌سازد و نیاز به به کار گیری یک نیروی انسانی صرفاً برای ساخت برنامه‌ی لازم برای طراحی صحنه‌های خلاقانه و چشمگیر را مرتفع می‌کند.
