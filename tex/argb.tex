\فصل{دیودهای نوری آدرس‌پذیر}
\برچسب{chap:argb}

دیودهای نوری آدرس‌پذیر، توان دریافتی از منبع تغذیه را به در خواست کنترلر به نور تبدیل می‌کنند؛ که آدرس‌پذیر بودن آن‌ها نشانگر توانایی نشان‌دادن چندین رنگ مختلف در آن واحد است. هر دیود یا گروه دیودها به یک مدار مجتمع مخصوص متصل است که با کنترل جریان مربوط به هر رنگ، طیف رنگ و میزان روشنایی را طبق فرمان کنترلر تعیین می‌کند. یک ریسه‌ی آدرس‌پذیر دیودهای نوری پیکسلی در تصویر \رجوع{fig:ws2811} دیده می‌شود.

\شروع{شکل}[ht]
\تنظیم‌ازوسط
\درج‌تصویر[پهنا=0.5\پهنای‌متن]{ws2811}
\شرح{یک ریسه‌ی \چر{LED} پیکسلی\مرجع{amzn:B01AG923GI}}
\برچسب{fig:ws2811}
\پایان{شکل}

تفاوت مهمی بین دیودهای نوری سه رنگ کنترل‌شونده‌ی معمولی و دیودهای آدرس‌پذیر وجود دارد. دیودهای نوع اول، از یک دیود سه رنگ و یا ۳ مجموعه دیودهای همرنگ به عنوان خروجی استفاده می‌کنند که به یک کنترلر متصل شده و در هر لحظه تنها می‌توان یک رنگ مشخص را فعال کرد و دیودهای همرنگ به یک میزان روشنایی دارند، به عنوان مثال می‌توان همه‌ی دیودهای قرمز و سبز را روشن نموده و نتیجه یک رنگ زرد خواهد بود؛ اما امکان روشن کردن برخی از دیود‌های سبز و خاموش یا کم‌نور نگه داشتن بقیه دیودهای سبز به صورت همزمان وجود ندارد و در نتیجه عملاً باید خروجی را-هر قدر بزرگ و یا گسترده باشد و از بخش‌های متعددی تشکیل شده باشد-به عنوان یک واحد در نظر گرفت.

اما، دیودهای آدرس‌پذیر این قابلیت را دارند که هر دیود و یا گروهی از دیود‌ها را به صورت مجزا رنگ‌دهی کرد. این آدرس‌دهی می‌تواند به صورت صریح یا ضمنی باشد؛ در حالت صریح هر گروه شناسه‌ای منحصر به فرد داشته و برای تخصیص رنگ به آن، فرمانی حاوی شناسه به ریسه ارسال می‌شود. در حالت ضمنی، هر گروه اولین فرمان تخصیص رنگ را دریافت کرده و پیام‌های بعدی را به همان صورتی که دریافت کرده به گروه بعدی مخابره می‌کند.

\قسمت{مدار مجتمع \چر{WS2811}}

همان‌طور که گفته شد، خروجی نوری سیستم متشکل از دیودهای نوری است. این دیودها در تعداد بالا (معمولا  ۵۰ یا ۱۰۰ دیود سه رنگ در هر ریسه) موجود بوده و ریسه‌ها قابلیت اتصال آبشاری به یکدیگر را دارند. ولتاژ تغذیه آن‌ها ۵ یا ۱۲ ولت است که بستگی به تکی یا گروهی بودن هر گروه همرنگ دارد. به دلیل تعداد به نسبت بالای دیودها نیاز است تا روشی برای فرمان‌پذیری از کنترلر به کار گرفته شود تا علاوه بر استفاده از سیم‌های ورودی کمتر، سرعت بالایی را برای نمایش و به روزرسانی رنگ هر دیود فراهم نماید. در ریسه‌های مبتنی بر مدار مجتمع \چر{WS2811}، ارتباط از طریق استفاده از یک سیم داده‌ی غیرهمزمان تحقق می‌پذیرد که پروتکل مخصوص خود را برای انتخاب رنگ دارد (شکل  \رجوع{fig:ws2811-wiring}). این مدار مجتمع توانایی انتخاب یک رنگ برای یک دیود و یا گروهی از دیودهای به هم متصل را دارد\مرجع{ws2811:datasheet}.

\شروع{شکل}[ht]
\تنظیم‌ازوسط
\درج‌تصویر[پهنا=0.7\پهنای‌متن]{ws2811-wiring}
\شرح{اتصالات الکتریکی مدار مجتمع \چر{WS2811}\مرجع{ws2811:datasheet}}
\برچسب{fig:ws2811-wiring}
\پایان{شکل}

از منظر الکتریکی، این قطعه یک منبع جریان سه کانال است که طبق پروتکل مخصوص خود، میزان جریان عبوری از هر کانال قرمز، سبز یا آبی را با دقت ۸ بیت تنظیم می‌کند. برای اتصال چند دیود به یک مدار مجتمع و تشکیل گروه‌های همرنگ، دیودهای با رنگ‌های یکسان با هم سری شده و جریان یکسانی از آن‌ها عبور می‌کند.

پروتکل مورد استفاده، یک سیگنال دیجیتال ۲۴ بیتی مدوله شده با پهنای پالس است که در آن همه‌ی بیت‌های مربوط به گروه از دیودها به یکباره فرستاده می‌شوند، پس از مدت زمان معینی سیگنال گروه دوم ارسال می‌شود و همین روند تا ارسال رنگ برای آخرین گروه تکرار می‌شود. پایه خروجی مدارات مجتمع هر گروه به صورت متوالی به ورودی گروه قبلی وصل می‌شود و ورودی اولین گروه از کنترلر دریافت می‌گردد. هر آی‌سی اولین گروه ۲۴ بیتی از داده‌ها را دریافت می‌کند و بقیه‌ی بیت‌ها را دست نخورده به آی‌سی بعدی منتقل می‌سازد\مرجع{ws2811:datasheet}. بدین ترتیب، از لحاظ نظری می‌توان یک توالی بی‌نهایت طولانی از گروه‌های دیودی ایجاد کرد.

البته این روش معایب خود را نیز دارد؛ از جمله محدودیت در سرعت تغییر رنگ کل ریسه که منجر به کاهش سرعت قابل ملاحظه‌ای در ریسه‌های بسیار بلند خواهد شد. علت این امر این است که هر مدار مجتمع تا وقتی اطلاعات رنگ مربوط به خود را دریافت نکند اقدام به باز ارسال بقیه اطلاعات نخواهد کرد و در نتیجه برای به‌روز رسانی هر گروه لازم است دوباره رنگ گروه‌های قبل از آن مجددا فرستاده شود. این محدودیت موجب می‌شود به‌روزرسانی یک به یک گروه‌های دور از ابتدای ریسه از لحاظ زمانی به صرفه نباشد و کندی قابل توجهی در نمایش نهایی ایجاد کند؛ پس بهتر است رنگ هر گروه ابتدا در حافظه‌ای ذخیره شود و در بازه‌های زمانی مشخصی کل ریسه به روز رسانی شود.

\قسمت{پروتکل برقراری ارتباط}

\شروع{شکل}[ht]
\تنظیم‌ازوسط
\درج‌تصویر[پهنا=0.9\پهنای‌متن]{ws2811-proto}
\شرح{پروتکل مورد استفاده در ریسه‌ی مبتنی بر تراشه‌ی  \چر{WS2811}\مرجع{ws2811:datasheet}}
\برچسب{fig:ws2811-proto}
\پایان{شکل}

برای برقراری ارتباط در این ماژول از روش مدولاسیون کد پالس (\واژه{pcm}) برای تعیین کردن رنگ هر گروه استفاده می‌شود\مرجع{ws2811:datasheet}. پی.سی.ام. روشی برای انتقال اطلاعات از یک وسیله به وسیله دیگر با استفاده از کدگذاری اطلاعات با استفاده از مدت روشن یا خاموش بودن یک خروجی دیجیتال در زمان‌هایی مشخص است. به دلیل استفاده از خروجی دیجیتال این روش برای بسیاری از ابزارهای توکار مانند میکروکنترلرها و با استفاده از ورودی و خروجی همه منظوره و یا به طرقی دیگر و با روش‌های سخت‌افزاری مناسب است.

برای تعیین کردن رنگ هر گروه نیاز به ۲۴ بیت اطلاعات است که به سه رنگ ۸ بیتی تقسیم شده و مقدار هر کدام از این هشت بیت نمایانگر رنگ یکی از سه کانال قرمز سبز و آبی است. این اطلاعات طبق پروتکل و به صورت پشت سرهم به اولین گروه در هر ریسه ارسال می شود (شکل \رجوع{fig:ws2811-proto}). مدار مجتمع موجود در گروه اول سپس این اطلاعات را دریافت کرده و اولین گروه بیست و چهار بیتی از آن را برای رنگ خود استفاده می کند. سپس بقیه اطلاعات را از طریق پایه خروجی خود که به ورودی گروه دوم متصل است مخابره می‌کند و این روند تا موقعی که گروه های ۲۴ بیتی به پایان برسد تا انتهای ریسه ادامه می یابد\مرجع{ws2811:datasheet}.

فرکانس مورد استفاده برای برقراری ارتباط در این پروتکل ۸۰۰ کیلوهرتز است\مرجع{ws2811:datasheet} اما بعضاً در برخی از ریسه‌ها که با همین پروتکل کار می‌کنند فرکانس ۴۰۰ کیلوهرتز نیز مشاهده شده است. با توجه به شرایط و ضوابط تعیین شده در برگه اطلاعات مدار مجتمع یاد شده، از لحاظ زمان بندی محدودیت های خاصی به روی کنترل قرار داده شده است و برای تحقق آنها نیاز است که زمان بندی دقیق رعایت شود\مرجع{ws2811:datasheet}. در کنترل استفاده شده به علت محدودیت مقدار فرکانس ساعت سیستم، پیاده کردن این پروتکل با استفاده از ورودی و خروجی همه منظوره به صورت تماماً نرم افزاری موجب اختصاص بخش زیادی از توان پردازشی به این کار و در نتیجه کاهش پاسخگویی کلی سیستم می‌شود؛ به همین جهت برای تطابق سیگنال کنترلی تولید شده با مشخصات ذکر شده در پروتکل یاد شده از واحد فرستنده و گیرنده غیرهمزمان فراگیر در حال صرفاً خروجی در تراشه پردازنده استفاده شد.

