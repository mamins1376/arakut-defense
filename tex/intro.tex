\فصل{مقدمه}
\شماره‌گذاری‌صفحه{arabic}‬

\برچسب{sec:intro}

در ابتدا و به هنگام شروع پروژه، اطلاعات کمی درباره‌ی سخت‌افزار قابل استفاده و همچنین نرم‌افزار مورد نیاز به دست آمد. به هنگام طراحی و پس از کمی تحقیق مشخص شد که روش‌های متعددی برای برنامه‌نویسی و تولید نرم‌افزار پروژه - چه \واژه{firmware} (نرم‌افزاری که روی سیستم توکار اجرا می‌شود و به صورت سنتی و از دید کاربر غیرقابل تغییر تلقی می‌گردد - در اینجا تراشه‌ی \چر{\چر{ESP8266}}) و چه برنامه کاربردی قابل اجرا بر روی دستگاه کاربر به عنوان کنترل - وجود دارد. به علت جستجوی ناکافی و یا به دلایلی نامعلوم، نرم‌افزارهای موجود در این زمینه از انظار پنهان ماند و در نتیجه با فرض این که نیاز به برنامه‌نویسی از پایه و صفر، هم برای کنترل (دستگاه کاربر) و هم برای کنترلر مبرم است، این کار شروع شد؛ غافل از اینکه وقت و انرژی‌ای که در راستای برنامه‌نویسی صرف خواهد شد استفاده‌ی چندانی در آینده نخواهد داشت - حداقل مستقیماً برای این پروژه. گرچه، یادگیری فراوانی از جزئیات، طرزکار و کاربردهای ممکن ماژول مورد استفاده حاصل آمد که سعی کرده‌ام دست‌کم بخشی از آن را در فصل مرتبط با تراشه‌ی اصلی (فصل \رجوع{chap:esp}) بگنجانم؛ امید است روزی مفید فایده قرار گیرد و در پروژه‌ی دیگری به کار آید.

در فصول پیش رو، خلاصه‌ای از اطلاعات زمینه‌ای برای این پروژه آورده شده است. در فصل \رجوع{chap:iot}، توضیحاتی در مورد \واژه{iot} ارائه شده است که بستر اصلی ارتباط بین کاربر و دستگاه است. در فصل \رجوع{chap:controller} نمونه‌هایی انواع کنترلرهایی که از تکنولوژی‌های بی‌سیم (و به صورت بخصوص در باند ۲٫۴ گیگاهرتز) استفاده می‌کنند مشاهده می‌شود و در فصل \رجوع{chap:esp}، اطلاعات مفصلی درباره‌ی تراشه‌ی اصلی استفاده شده در کنترلر این پروژه که بخشی از آن نتیجه‌ی پرسه‌های طولانی مدت در اعماق فضای اینترنت و غالباً برای کنجکاوی و بعضاً به جهت نیاز به دانستن دقیق نحوه‌ی کارکرد این وسیله انجام شده است می‌باشد.

بالاخره فصل \رجوع{chap:setup} چگونگی کارکرد سیستمی که به طور خاص در این پروژه ساخته شده و نرم‌افزارهایی که در آن به کار رفته می‌پردازد و در نهایت، فصل \رجوع{chap:wrapup} نتیجه‌گیری حاصل از این پروژه را مستند می‌سازد.

