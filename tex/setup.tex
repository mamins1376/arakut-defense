\فصل{روند ساخت و امکانات پروژه}
\برچسب{chap:setup}

در این فصل بر مبنای زمینه‌ی ارائه شده درباره‌ی \واژه{iot}، ماژول \واژه{wifi}، مجموعه‌ی موتورها، درایور آنها و سیستم ساخته شده بررسی و تشریح می‌گردد.

\شروع{شکل}[ht]
\تنظیم‌ازوسط
\درج‌تصویر[پهنا=0.9\پهنای‌سطر]{setup}
\شرح{سیستم ساخته شده}
\برچسب{fig:setup}
\پایان{شکل}

\قسمت{نحوه کار و دیاگرام سیستم}

کاربر با تلفن‌هوشمند، رایانه‌لوحی و یا رایانه شخصی یا قابل حمل خود به کنترلر یا نقطه دسترسی \واژه{wifi} متصل می‌شود؛ گرچه لزومی به وجود دستگاه اخیر نیست زیرا کنترلر نیز می‌تواند خود به عنوان یک نقطه دسترسی مستقل عمل کند. سپس کاربر می‌تواند سرعت و جهت دلخواه خود را تنظیم کند.

\شروع{شکل}[ht]
\تنظیم‌ازوسط
\درج‌تصویر[پهنا=0.4\پهنای‌متن]{diagram}
\شرح{بلوک دیاگرام سیستم}
\برچسب{fig:diagram}
\پایان{شکل}

بلوک‌های تشکیل‌دهنده‌ی سیستم در شکل \رجوع{fig:diagram} دیده می‌شود. این سیستم‌ها از چهار قسمت کنترل از راه دور، درایور، موتورها، منبع تغذیه و کنترلر یا راه‌انداز تشکیل شده‌اند؛ که امکان ادغام دو یا سه مورد آخر به عنوان یک واحد مستقل وجود دارد.

\شروع{شکل}[ht]
\تنظیم‌ازوسط
\درج‌تصویر[پهنا=0.9\پهنای‌سطر]{close-up}
\شرح{نمای نزدیک از ماژول وای-فای}
\برچسب{fig:close-up}
\پایان{شکل}

\شروع{شکل}[ht]
\تنظیم‌ازوسط
\درج‌تصویر[scale=0.6]{app}
\شرح{برنامه‌ی مبتنی بر وب}
\برچسب{fig:wled}
\پایان{شکل}


\شروع{شکل}[ht]
\تنظیم‌ازوسط
\درج‌تصویر[پهنا=\پهنای‌سطر]{1678102218129}
\شرح{برد ‌‌BluePill}
\برچسب{fig:bluepill}
\پایان{شکل}


\قسمت{نرم‌افزار به کار رفته}

در حال حاضر روش‌ها و ابزارها و کتاب‌خانه‌های نرم‌افزاری متعددی برای برنامه‌نویسی این تراشه وجود دارد. به علت سادگی، کنترل بیشتر و البته ترجیحات شخصی، زبان برنامه‌نویسی C و سیستم \واژه{pio} انتخاب شد. این مجموعه یک \واژه{sbs} است که مدیریت و نصب کتابخانه‌های نرم‌افزاری را ساده‌تر می‌کند.


