\فصل{روند ساخت و امکانات پروژه}
\برچسب{chap:setup}

در این فصل بر مبنای زمینه‌ی ارائه شده درباره‌ی \واژه{iot}، ماژول \واژه{wifi} و ریسه‌ی دیودهای نوری پیکسلی، سیستم ساخته شده بررسی و تشریح می‌گردد.

\شروع{شکل}[ht]
\تنظیم‌ازوسط
\درج‌تصویر[پهنا=0.9\پهنای‌سطر]{setup}
\شرح{سیستم ساخته شده}
\برچسب{fig:setup}
\پایان{شکل}

\قسمت{نحوه کار و دیاگرام سیستم}

کاربر با تلفن‌هوشمند، رایانه‌لوحی و یا رایانه شخصی یا قابل حمل خود به کنترلر یا نقطه دسترسی \واژه{wifi} متصل می‌شود؛ گرچه لزومی به وجود دستگاه اخیر نیست زیرا کنترلر نیز می‌تواند خود به عنوان یک نقطه دسترسی مستقل عمل کند. پس از اعمال تنظیمات مربوطه و ارائه‌ی مشخصات دیودهای وصل شده، کاربر می‌تواند نور دلخواه خود را تنظیم کند.

\شروع{شکل}[ht]
\تنظیم‌ازوسط
\درج‌تصویر[پهنا=0.4\پهنای‌متن]{diagram}
\شرح{بلوک دیاگرام سیستم}
\برچسب{fig:diagram}
\پایان{شکل}

بلوک‌های تشکیل‌دهنده‌ی سیستم در شکل \رجوع{fig:diagram} دیده می‌شود. این سیستم‌ها از چهار قسمت کنترل از راه دور، خروجی نوری، منبع تغذیه و کنترلر یا راه‌انداز تشکیل شده‌اند؛ که امکان ادغام دو یا سه مورد آخر به عنوان یک واحد مستقل وجود دارد. خروجی نوری ممکن است فقط یک دیود نوری چند رنگ باشد یا آرایه‌ای از دیودهای مجزا یا بلوک‌های تک‌رنگ باشد. به همین صورت کنترلرها ممکن است تک‌رنگ باشند یا توانایی اختصاص رنگ به هر دیود یا گروه دیودها را داشته باشد. دیودهایی که به صورت گروهی متصل شده‌اند ممکن است سری یا موازی باشند که این انتخاب تعیین کننده‌ی ولتاژ تغذیه مورد نیاز برای راه‌اندازی آنهاست. همچنین تعداد دیودها مشخص‌کننده‌ی جریان و طبعا توان مصرفی است که توقع می‌رود منبع تغذیه قادر به تأمین آن باشد.

\قسمت{پیاده‌سازی}

سیستم ساخته شده‌ در شکل \رجوع{fig:setup} مشاهده می‌شود. جعبه‌ی سمت راست تصویر یک منبع تغذیه‌ی ۵ ولتی را نشان می‌دهد که توانایی لازم برای تهیه توان مصرفی دستگاه را دارد. در سمت چپ تصویر ریسه‌ی دیودها دیده می‌شود و شامل ۵۰ دیود ۳ رنگ کنترل شونده با پروتکل \چر{WS2811} است. بردی که این دو قسمت را به یکدیگر مرتبط می‌کند حاوی یک ماژول \واژه{wifi} با مدار مجتمع \چر{\چر{ESP8266}}‌ است که یک پردازنده مبتنی بر معماری \چر{Xtensa} بوده و ادوات لازم برای برقراری ارتباط با نقطه‌ی دسترسی \واژه{wifi} یا ایجاد آن برای کاربر را دارد\مرجع{esp8266}.

\شروع{شکل}[ht]
\تنظیم‌ازوسط
\درج‌تصویر[پهنا=0.9\پهنای‌سطر]{close-up}
\شرح{نمای نزدیک از کنترلر}
\برچسب{fig:close-up}
\پایان{شکل}

در شکل \رجوع{fig:close-up} نمای کنترلر ساخته شده از نزدیک دیده می‌شود. در وسط تصویر ماژول \واژه{wifi} قابل برنامه‌ریزی مشاهده می‌شود که سیم سفید خروجی آن بوده و به ریسه متصل می‌گردد. دو سیم مشکی و زرد پایین تصویر ورودی ۵ ولت است که برای تغذیه‌ی ریسه و ماژول کاربرد دارد، گرچه این ماژول نیاز به ولتاژ ۳٫۳ ولت دارد و مدار مجتمع سمت چپ تصویر که یک مبدل ولتاژ خطی است این تبدیل را انجام می‌دهد. دو کلید واقع در سمت چپ برای شروع مجدد برنامه و فعال کردن حالت برنامه‌ریزی استفاده می‌شوند.

\قسمت{راه‌اندازی}
اولین باری که کنترلر روشن می‌شود، نقطه دسترسی را برای اتصال مستقیم کنترل کاربر فراهم می‌کند. در صورت تمایل کاربر می‌تواند مشخصات نقطه دسترسی دیگری را (که می‌تواند یک مسیریاب شبکه یا کنترلر دیگری هم باشد) وارد کند تا اتصال بین کاربر و کنترلر بر آن بستر باشد.

مزیت اتصال به نقطه دسترسی مجزا در فراهم آوردن امکان اتصال همزمان یک کاربر به چندین کنترلر است که می‌تواند موجب گسترش محدوده‌ی پوشش خروجی‌ها، توزیع بهینه‌تر مصرف توان (استفاده از چندین منبع تغذیه کم توان‌تر به جای یک منبع پرتوان)، بیشتر کردن برد شبکه و در نتیجه عدم نیاز به نقطه دسترسی بلند-برد شود.

پس از راه‌اندازی بستر شبکه، کاربر یک از دو برنامه کاربردی مربوط به کنترلر را استفاده می‌کند. برنامه اول یک برنامه تحت وب است که نیازی به نصب نداشته و مستقیماً از هر مرورگر وب به‌روزی قابل استفاده است، گرچه امکانات تعبیه‌شده در آن به نسبت برنامه‌ی دوم کمتر است. شمایی از این برنامه در شکل \رجوع{fig:wled}‌ مشاهده می‌شود. در این صفحه به کمک دایره‌ی مرکزی می‌توان رنگ دیودها را تنظیم کرد و یا رنگ‌های از پیش انتخاب شده‌ای را فعال کرد.

\شروع{شکل}[ht]
\تنظیم‌ازوسط
\درج‌تصویر[scale=0.6]{wled}
\شرح{برنامه‌ی مبتنی بر وب}
\برچسب{fig:wled}
\پایان{شکل}

برنامه دوم مخصوص سیستم‌عامل کاربر بوده و پس از نصب قابل استفاده است. چون در حال حاضر سیستم‌عامل‌ها به برنامه‌ی بومی خود اختیارات بسیار بیشتری نسبت به برنامه‌های تحت وب ارائه می‌دهند، طبیعتاً محدودیت‌های حاضر در پلتفرم وب رفع شده و امکان دسترسی به فضای بیشتر، تعامل با دیگر برنامه‌ها (برای اشتراک گذاری نورپردازی طراحی شده)، استفاده‌ی مستقیم از صدای پخش شده توسط سیستم‌عامل و بهره‌گرفتن از آن به عنوان منبع صدا (که برای ایجاد خودکار نورپردازی‌های مختلف و زنده کاربرد دارد) فراهم می‌گردد. بدین صورت کاربر می‌تواند برای هماهنگی نورپردازی با موسیقی، پس از راه‌اندازی اولیه‌ي برنامه‌ی مخصوص سیستم‌عامل از برنامه‌ی دلخواه خود برای پخش صدا استفاده کند؛ چه موسیقی، چه فیلم و چه هر رسانه‌ی دیگری (شکل \رجوع{fig:ledfx}).

برای هماهنگ‌سازی نورپردازی با موسیقی و ایجاد جلوه‌های جذاب بصری می‌توان از دو روش به تنهایی و یا به صورت ترکیبی استفاده نمود. در روش دستی، نوع مختلف جلوه‌ها توسط تدوین‌گر و با نرم‌افزار مخصوص طراحی و آماده می‌شود. از مزایای این روش می‌توان به انعطاف و اختیار تام در انتخاب جلوه‌ها و زمان‌بندی آن‌ها، و همچنین استفاده‌ی مجدد از جلوه‌ها، و نیاز به توان پردازشی بسیار کم اشاره کرد؛ اما از معایب این روش وقت‌بر بودن و نیاز به نیروی انسانی جهت برنامه‌ریزی است.

روش دیگر، استخراج اطلاعات لازم از موسیقی و ایجاد یک فضای چند بعدی از پارامترهای لحظه‌ای مربوطه (مانند بلندی صدا، طنین صدا و یا استخراج صدای خواننده یا یک ساز خاص) و سپس نگاشت آن‌ها به ابعاد مختلف جلوه‌های نوری روشنایی، رنگ یا سرعت پخش جلوه و در نهایت انتخاب از میان جلوه‌ها و پخش نتایج در خروجی است؛ چه به صورت لحظه‌ای و در زمان واقعی و چه به صورت از پیش پردازش شده. خوبی این روش در خودکار بودن آن است و پس از طراحی الگوریتم جلوه، دیگر نیازی به تدوین‌گر نیست. البته در صورتی که اجرای الگوریتم در زمان واقعی باشد نیاز به پردازنده‌ی مناسب با سرعت لازم وجود دارد. نمونه‌ای از چنین کنترلری وجود دارد که به کاربر امکان می‌دهد حالت دلخواه خود را از میان تعداد از پیش برنامه‌ریزی شده انتخاب کند\مرجع{amzn:B0792T73VB}. همچنین، مدل دیگری که امکان برنامه‌ریزی جلوه‌های دلخواه بصری را به کاربر ارائه می‌دهد نیز در دسترس است\مرجع{amzn:B01M4IX5FF}.

\شروع{شکل}[ht]
\تنظیم‌ازوسط
\درج‌تصویر[پهنا=\پهنای‌سطر]{ledfx}
\شرح{رابط برنامه‌ی بومی قابل نصب بر رایانه}
\برچسب{fig:ledfx}
\پایان{شکل}

\قسمت{نرم‌افزار به کار رفته}

سه واحد نرم‌افزاری مستقل در طول پروژه به کار رفته است. یمی نرم‌افزار اولیه بود که در ابتدا توسعه داده شد و پس از حصول نمونه‌ی اولیه‌ای که کار می‌کرد اما امکانات آن بسیار محدود بود و در واقع تقریبی از پروژه‌ی WLED بود که بعداً با آن عوض شد. واحد دوم خود پروژه‌ی WLED است و واحد سوم پروژه‌ی \چر{LedFx}.

\زیرقسمت{نرم‌افزار اولیه}

در حال حاضر روش‌ها و ابزارها و کتاب‌خانه‌های نرم‌افزاری متعددی برای برنامه‌نویسی این تراشه وجود دارد. به علت سادگی، کنترل بیشتر و البته ترجیحات شخصی، زبان برنامه‌نویسی C و سیستم \واژه{pio} انتخاب شد. این مجموعه یک \واژه{sbs} است که مدیریت و نصب کتابخانه‌های نرم‌افزاری را ساده‌تر می‌کند.

قبل از این که این نرم‌افزار با WLED جایگزین شود به مرحله‌ای رسیده بود که توانایی برقراری ارتباط با ریسه‌ی LED را داشت و توانایی تنظیم کل ریسه به یک رنگ دلخواه از طریق رابط کاربری مبتنی بر وب پیاده‌سازی شده بود.

\زیرقسمت{پروژه‌ی \چر{WLED}}

در طی روند پروژه و چندین ماه پس از شروع آن، مشخص شد که جامعه‌ای از کاربران حول پروژه‌ای \واژه{opensource} به نام دابلیو.ال.ای.دی\پانویس{WLED} وجود دارد (برای مراجعه به وبگاه آن،‌ مرجع \مرجع{wled} را ببینید). این پروژه‌ی نرم‌افزاری دارای امکانات فراوانی بود که از توان یک نفر برای برنامه‌نویسی بسیار فراتر می‌رود. از امکانات این پروژه می‌توان به پشتیبانی بسیار گسترده‌تر و متنوع‌تر از پروتکل‌های مختلف ریسه‌های \چر{LED}، امکانات راه‌اندازی و تنظیم اولیه‌ی \واژه{wifi} بسیار آسان‌تر و توانایی آن در کنترل ریسه به شیوه‌های مختلف اشاره کرد. طرزکار این برنامه با برنامه‌ی اولیه‌ی توسعه داده شده تفاوت بنیادینی ندارد و از همان روش برای ارسال فرامین و دریافت پاسخ بین مرورگر و ماژول \واژه{wifi} استفاده می‌کند.

این نرم‌افزار قادر است تا به دو روش اصلی ریسه را کنترل کند؛ روش اول استفاده از رابط کاربری اصلی نرم‌افزار که در هنگام راه‌اندازی در مرورگر کاربر بارگزاری می‌شود است که ساده‌تر بوده و نیازی به نصب ندارد (شکل \رجوع{fig:wled}). در این قسمت می‌توان جلوه‌های از قبل تعریف شده را فعال کرد که بدون نیاز به اتصال مداوم به کاربر و با انجام پردازش‌های مورد نیاز روی خود کنترلر نمایش داده می‌شوند. تعداد افکت‌های موجود و قابلیت شخصی‌سازی و اعمال تنظیمات دلخواه در این صفحه، خود نمایانگر توانایی این برنامه در ایجاد صحنه‌های متنوع و همچنین حجم زیاد وقتی که برای طراحی آن‌ها صرف شده، است.

در روش دوم، WLED به کاربر خود این امکان را می‌دهد که با فرستادن فرمان‌هایی بر روی شبکه، رنگ هر دیود را به صورت مجزا مشخص کند. در این روش پس از آماده‌سازی و انتخاب نوع و طول ریسه، بر روی پورت ۱۹۴۴۶ پروتکل لایه‌ی انتقال یو.دی.پی (\واژه{udp})، رنگ و شماره‌ی هر دیود فرستاده می‌شود و بدین طریق امکان کنترل مستقیم و از راه دور ریسه بر بستر شبکه و با تکنولوژی \واژه{iot} محقق می‌گردد\مرجع{wled-udp}. در ادامه خواهیم دید که نرم‌افزار اجرا شده روی سیستم کاربر، چگونه از این قابلیت برای ایجاد نورپردازی هماهنگ با صوت و موسیقی استفاده می‌کند.

\زیرقسمت{پروژه‌ی \چر{LedFx}}

این پروژه‌ی نرم‌افزاری، جهت دریافت صدای در حال پخش از رایانه‌ی مجهز به یکی از سیستم‌عامل‌های مایکروسافت ویندوز\پانویس{Microsoft Windows}، اپل مک‌اواس\پانویس{Apple Mac OS} و یا گنو\textbackslash{لینوکس}\پانویس{GNU/Linux} و ارسال آن به یکی از خروجی‌های سازگار به کار می‌رود (برای مشاهده‌ی وبگاه برنامه به \مرجع{ledfx} مراجعه کنید). در پروژه‌ی حاضر ارتباط بین LedFx و WLED از طریق پورت یاد شده در بالا بر روی شبکه استفاده می‌شود.


