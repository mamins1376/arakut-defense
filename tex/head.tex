\محیط‌نو{توپر}{\شمایل‌سیاه}{}

\سبک‌این‌صفحه{empty}
\شماره‌گذاری‌صفحه{alph}‬

\شروع{وسط‌چین}

\onehalfspacing

\شروع{تیتر}
\فضای‌و{-15mm}
\درج‌تصویر{arakut}

\departfa

\فضای‌و{2cm} پایان‌نامه پروژه دوره کارشناسی مهندسی برق گرایش کنترل

\فضای‌و{2cm}\درشت‌تر\titlefa

\فضای‌و{2.5cm}\درشت\authorfa

\فضای‌و{3cm}\اندازه‌عادی استاد راهنما:\بند\درشت \supervisorfa

\فضای‌و{3cm}\اندازه‌عادی\datefa

\پایان{تیتر}

\صفحه‌پاک
\سبک‌این‌صفحه{empty}

\شروع{توپر}

\فضای‌و*{15mm}\درشت‌درشت\titlefa

\فضای‌و{2cm}\اندازه‌عادی توسط:\بند\درشت‌تر\authorfa

\فضای‌و{2cm}\درشت پایان‌نامه\بند ارائه شده به عنوان دفاعیه پروژه پایانی برای اخذ درجه کارشناسی\بند در رشته مهندسی برق\بند از\بند \instfa\بند اراک - ایران

\doublespacing

\فضای‌و{1cm}\اندازه‌عادی ارزیابی و تصویب شده
توسط\پرنقطه‌ا{} در تاریخ\پرنقطه‌ا{} و نمره\پرنقطه‌ا{} با درجه\پرنقطه‌ا{}

\supervisorfa
(استاد راهنما) \پرنقطه‌ا
استاد \instfa

دکتر فربد ستوده
(استاد داور) \پرنقطه‌ا
استاد \instfa

\پایان{توپر}

\صفحه‌پاک
\سبک‌این‌صفحه{empty}

\مقداربعد{\fboxrule}{2pt}
\مقداربعد{\fboxsep}{3pt}
\کادربا{%
	\شروع{صفحه‌کوچک}[t]{0.12\پهنای‌متن}%
		\فضای‌و{0pt}
		\درج‌تصویر[پهنا=\پهنای‌متن]{arakut}%
	\پایان{صفحه‌کوچک}%
	\شروع{صفحه‌کوچک}[t]{0.7\پهنای‌متن}%
		\تنظیم‌ازوسط
		\فضای‌و{4pt}
		\شروع{توپر}
			بسمه تعالی\بند
			فرم تعهد اصالت اثر\بند
			\فضای‌و{4pt} \departfa
		\پایان{توپر}
	\پایان{صفحه‌کوچک}%
	\شروع{صفحه‌کوچک}[t]{0.13\پهنای‌متن}%
		\فضای‌و{4pt}
		\مقداربعد{\ضخامت‌کادربا}{1pt}%
		\کادربا{فرم \چر{F-16}}\\تاریخ:\\شماره:
	\پایان{صفحه‌کوچک}
}

\فضای‌و{1cm}
\شروع{صفحه‌کوچک}[t]{0.85\پهنای‌متن}
\onehalfspacing
اینجانب \authorfa{} دانشجوی دوره کارشناسی رشته مهندسی برق با شماره دانشجویی ۹۵۱۰۱۹۰۴۵ متعهد می شوم که مطالب مندرج در این پايان نامه حاصل کار پژوهشی اینجانب تحت نظارت و راهنمایی اساتید \instfa بوده و به دستاوردهای دیگران که در این پژوهش از آنها استفاده شده است، مطابق مقررات و روال متعارف، ارجاع و در فهرست منابع و مآخذ ذکر گردیده است. این پايان نامه  قبلا برای احراز هیچ مدرک هم سطح یا بالاتر ارائه نگردیده است. در صورت اثبات تخلف در هر زمان، مدرک تحصیلی صادر شده توسط دانشگاه از درجه اعتبار ساقط بوده و دانشگاه حق پیگیری قانونی خواهد داشت.

کلیه نتایج و حقوق حاصل از این پايان نامه متعلق به \instfa است و هرگونه استفاده از نتایج علمی، واگذاری اطلاعات به دیگران، چاپ یا تکثیر، نسخه برداری، ترجمه و اقتباس از آن بدون موافقت کتبی \instfa ممنوع است. نقل مطالب با ذکر مآخذ بلامانع است.

\فضای‌و{1cm}
\متن‌سیاه{عنوان پایان‌نامه}: \titlefa

\فضای‌و{1cm}
\متن‌سیاه{استاد راهنما}:
\supervisorfa
\پایان{صفحه‌کوچک}


\فضای‌و{1.5cm}\authorfa\بند\متن‌سیاه{امضا}

\صفحه‌پاک
\سبک‌این‌صفحه{empty}

\شروع{تیتر}
\درشت تقدیم به کسانی که بیان حقیقت سرچشمه شور آنهاست.

\صفحه‌پاک
\سبک‌این‌صفحه{empty}

\اندازه‌عادی تشکر و قدردانی

\پایان{تیتر}
\پایان{وسط‌چین}

وجود خودم را وامدار پدر و مادرم و تحمل‌های مداوم و بی‌دریغ آن‌ها هستم. هر تلاشی برای توصیف زحمات ایشان محکوم به خطاست؛ بنابراین به جمله‌ای کوتاه بسنده می‌کنم تا دست‌کم ناچیز بودن آن مشهود باشد: از شما ممنونم.

لازم می‌دانم از استاد گرانقدرم، جناب آقای \supervisorfa، در وهله‌ی اول به خاطر حق استادی که به گردن من دارند و سپس به خاطر قابل دانستن اینجانب برای انجام این پروژه و همچنین پیگیری‌های مستمر و صبر مثال‌زدنی ایشان در طول مدت انجام آن، کمال تشکر را داشته باشم. بدون شک، ایشان نقش بسیار پررنگی در شکل‌گیری دید من نسبت به علم مخابرات و به طور کلی در حوزه‌ی مهندسی برق داشته‌اند و دارند؛ باشد که بتوانم بخشی از این دین را ادا کنم و از بپیمودن این راه باز ناایستم.

از جناب آقای دکتر فربد ستوده که زحمت داوری و نقد این پایان‌نامه را تقبل نمودند کمال امتنان را دارم. اشتیاق ایشان برای آموزاندن مباحث روز حوزه تخصصشان همواره برای من ستودنی بوده است.

همچنین، انجام این پروژه منوط به وجود بستر نرم‌افزاری مناسب بود که به لطف کاربران و توسعه‌دهندگان آن، بهره‌گیری از آن برای عموم آزاد است. از تمام کسانی که به نحوی به تکامل این مجموعه کمک کردند سپاسگزارم.

\فضای‌و{15mm}
\پررا
\شروع{صفحه‌کوچک}[t]{0.3\پهنای‌متن}
\authorfa\par\datefa
\پایان{صفحه‌کوچک}

\صفحه‌پاک
\بیفزاسطرفهرست{toc}{chapter}{چکیده}

\شروع{وسط‌چین}
\قلم‌تیتر{چکیده}
\پایان{وسط‌چین}

\واژه‌متن{iot} در حال گسترش است و سیستم‌های کنترلی هوشمند با استفاده از این تکنولوژی می‌توانند به کاربران توانایی خلق مجموعه‌های جذاب را با هزینه‌ی کم و دسترسی‌پذیری بالا، بدون نیاز به دانش فنی پیچیده ارائه دهند. این مجموعه‌ نیازمند مرورگر وب، کنترلر، موتور \lr{DC} و منبع توان است که فرآیند ساخت یک نمونه از آن در متن پیش رو مستند گردیده است. \واژه‌متن{iot} برای برقراری ارتباط بین کاربر که از طریق مرورگر وب یا نرم‌افزار بومی استفاده می‌کند با کنترلر که قسمت اصلی آن یک تراشه‌ی \چر{ESP8266} است و به ادوات \واژه‌متن{wifi} برای اتصال به شبکه \واژه‌متن{ip} و یک پردازنده مجهز است به کار می‌رود تا اطلاعات لازم از طریق شبکه محلی یا شبکه اینترنت منتقل شوند. کنترلر مستقیماً از طریق رابط سریال میکرکنترلر با آن در ارتباط است و منبع تغذیه توان مورد نیاز برای این دو را فراهم می‌سازد.

\فضای‌و{6mm}
\متن‌سیاه{کلیدواژه}:
\واژه‌متن{iot}،
\چر{ESP8266}،
\چر{DC Motor}،
\چر{L911OS}

